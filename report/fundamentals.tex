\section{Fundamental Graph Routing}
\label{fundamental}

In this report I will use the following terms and notations

\defi{In this thesis I will use the term \emph{node} when referring to a device that can both transmit and receive information, execute a routing algorithm based on the received message, and which is the part of the \manet.}
\defi{Given two nodes $n_1$ and $n_2$ in the plane. Then $\overline{n_1n_2}$ will indicate that there is a line-segment between node $n_1$ and $n_2$.}
\defi{Let $x$ be a line-segment, then $|x|$ is the length of $x$ in the space that is it. If no space is explicitly mentioned, this is the Euclidean space.}
\defi{A \emph{source} will be the node in the network that begins a transmission (by sending a message) and a \emph{sink(s)} will be the node(s) that is supposed to receive the message.}
\defi{The \emph{Routing Graph} is the graph that a message has to navigate through in order to go from the source to the sink. In most cases this graph is not unique.}
\defi{\emph{Geographic movement} is when a node changes its position, this may or may not affect the topology of the routing graph}
\defi{\emph{Topological movement} is when the movement in the node causes a change in the routing}
\defi{When a node $n_x$ can transmit information to another node $n_y$ (i.e. $n_y$ lies within the range of $n_x$ transmitter), then I will say that $n_x$ is \emph{adjacent} to $n_y$}
\defi{Given two nodes $n_x$ and $n_y$, I will say that there exists a \emph{bidirectional link} between the nodes, iff $n_x$ is adjacent to $n_y$ and $n_y$ is adjacent to $n_x$.}
\defi{When 2 nodes have a bidirectional link, and they are connected in the underlying routing graph, I will say that the two nodes are \emph{neighbours}.}
