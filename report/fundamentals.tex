\section{Fundamental Graph Routing}
\label{fundamental}
In this section I will describe the fundamentals of graph routing   . Furthermore I will use this section to introduce definitions and terminology that will be used throughout the thesis.

In order to be able to reason about \manet, we will have to give it a proper theoretical formulation. The most logical of these is that of the graph. A graph $G$ is a set of \emph{nodes} $V$ and \emph{edges} $E$ that connects the points. The nodes are represented as points, while the edges are represented as line-segments, with their endpoints being 2 different nodes. For simplicities sake, I will assume that the graph exists in the plane. I find this to be a reasonable simplification since almost all the literature about \manet and \anet that I have been able to find does the same, but also because there in practice shouldn't be a large difference between 2D and 3D networks. 

The object of routing in any network is to send a message from its start position, or its \emph{source}, to some end position, or \emph{sink}. In some cases there may be multiple sources and sinks, but in this thesis I will assume there is only one of each. I will now explain how I have chosen to model this network. 

In a graph I will say that a when a node $n_x$ can transmit information to another node $n_y$ (i.e. $n_y$ lies within the range of $n_x$ transmitter), then $n_x$ is \emph{adjacent} to $n_y$. However, having node $n_x$ being adjacent to $n_y$ does not necessarily mean that the two nodes can communicate\footnote{Two-way communication is important, since $n_x$ may ask $n_y$ to fetch it a document, but $n_y$ will be unable to send it to $n_x$. A possible solution could be to transmit the file through a different path to $n_x$, but I feel this would be too complex.}, since we do not know whether or not node $n_x$'s transmitter is more powerful than $n_y$'s (since transmitter devices in the real world are heterogeneous). I will therefore say that two nodes nodes $n_x$ and $n_y$ can only communicate if there exists a \emph{bidirectional link} between the nodes, and that there is only a bidirectional link iff $n_x$ is adjacent to $n_y$ and $n_y$ is adjacent to $n_x$.

\tikfig{connec-route}{Figure~\ref{fig:nodes} shows a set of nodes. The circle indicates the range that node a can transmit to -- all the other nodes have the same range, but they have been left out for clarity. Figure~\ref{fig:connec} shows us a connectivity graph. Figure~\ref{fig:routing_not_planar} and \ref{fig:routing_planar} are examples of routing graph, a subset of the connectivity graph. Since figure~\ref{fig:routing_planar} is planar, it would be preferred over figure~\ref{fig:routing_not_planar}.}

Now we can begin to define the \emph{connectivity graph}. Given a set of nodes $V$ the connectivity graph C will be $C = (V, E)$, where $E$ contains an edge between node $n_x$ and $n_y$ iff there is a bidirectional link between $n_x$ and $n_y$ -- see figure~\ref{fig:connec} for an example of this. It is important to note that the connectivity graph only tells us which nodes can communicate with each other in a \manet, not which actually does (since it may be impractical). Therefore I introduce the \emph{Routing Graph} R, where $R = (V, E\prime)$, where $E\prime \subseteq E$. the graph that a message has to navigate through in order to go from the source to the sink. In most cases this graph is not unique - as we can .




 











\defi{\emph{Geographic movement} is when a node changes its position, this may or may not affect the topology of the routing graph}
\defi{\emph{Topological movement} is when the movement in the node causes a change in the routing}


\defi{Given two nodes $n_1$ and $n_2$ in the plane. Then $\overline{n_1n_2}$ will indicate that there is a line-segment between node $n_1$ and $n_2$.}
\defi{Let $x$ be a line-segment, then $|x|$ is the length of $x$ in the space that is it. If no space is explicitly mentioned, this is the Euclidean space.}
