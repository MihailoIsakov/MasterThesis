\section{Fundamental Graph Routing}
\label{fundamental}
In this section I will describe the fundamentals of graph routing   . Furthermore I will use this section to introduce definitions and termanlogy that will be used throughout the thesis.

In order to be able to reason about \manet, we will have to give it a proper theoretical formulation. The most logical of these is that of the graph. A graph $G$ is a set of \emph{nodes} $V$ and \emph{edges} $E$ that connects the points. The nodes are represented as points, while the edges are represented as line-segments, with their endpoints being 2 different nodes. For simpilcities sake, I will assume that the graph exists in the plane. I find this to be a reasonable simplification since almost all the literature about \manet and \anet that I have been able to find does the same, but also because there in practice shouldn't be a large difference between 2D and 3D networks. 

In a graph I will say that a when a node $n_x$ can transmit information to another node $n_y$ (i.e. $n_y$ lies within the range of $n_x$ transmitter), then $n_x$ is \emph{adjacent} to $n_y$. However, having node $n_x$ being adjacent to $n_y$ does not necesarily mean that the two nodes can communicate\footnote{Two-way communication is important, since $n_x$ may ask $n_y$ to fetch it a document, but $n_y$ will be unable to send it to $n_x$. A possible solution could be to transmit the file through a different path to $n_x$, but I feel this would be too complex.}, since we do not know whether or not node $n_x$'s transmitter is more powerful than $n_y$'s.   

\tikfig{connec-route}{Figure~\ref{fig:nodes} shows the nodes of the network. The circle indicates the range that node a can transmit to (all the other nodes have the same range, but they have been left out for clarity) . Figure~\ref{fig:connec} shows us a connectivity grap. Figure~\ref{fig:routing} is the routing graph, a subset of the connectivity graph, giving us the Peterson graph.}



\defi{The \emph{Routing Graph} is the graph that a message has to navigate through in order to go from the source to the sink. In most cases this graph is not unique.}




 
\defi{Given two nodes $n_x$ and $n_y$, I will say that there exists a \emph{bidirectional link} between the nodes, iff $n_x$ is adjacent to $n_y$ and $n_y$ is adjacent to $n_x$.}
\defi{When 2 nodes have a bidirectional link, and they are connected in the underlying routing graph, I will say that the two nodes are \emph{neighbours}.}



\defi{A \emph{source} will be the node in the network that begins a transmission (by sending a message) and a \emph{sink(s)} will be the node(s) that is supposed to receive the message.}






\defi{\emph{Geographic movement} is when a node changes its position, this may or may not affect the topology of the routing graph}
\defi{\emph{Topological movement} is when the movement in the node causes a change in the routing}


\defi{Given two nodes $n_1$ and $n_2$ in the plane. Then $\overline{n_1n_2}$ will indicate that there is a line-segment between node $n_1$ and $n_2$.}
\defi{Let $x$ be a line-segment, then $|x|$ is the length of $x$ in the space that is it. If no space is explicitly mentioned, this is the Euclidean space.}
