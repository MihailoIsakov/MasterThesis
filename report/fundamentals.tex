\section{Fundamental Graph Routing}
\label{section:fundamental_graph_routing}
In this section I will describe the fundamentals of graph routing and introduce definitions and terminology that I will use throughout this thesis.

\subsection{Graph concepts}
\label{section:graph_concepts}
In order to be able to reason about \ac{manet}, we will have to model it. To do this I will use the graph. A graph $G$ is a set of \emph{nodes} $V$ and set of \emph{edges} $E$. An edge may have a cost to use -- often refereed to as the edge's \emph{weight}. If all edges have the cost of 1, the graph is said to be the \emph{unit} graph. The nodes are represented as points, while the edges are represented as line-segments, that uses two points as its end-points. In this thesis I will only work with graphs that exist in the plane. I find this to be a reasonable simplification, since it makes the graphs far simpler to reason about, reflects most of its uses, and because almost all the literature about \acp{manet} and \acp{anet} makes the same simplification. Furthermore I will define \emph{face} as being an area of the plane that is bounded by edges, and the \emph{outer face} as being the face that is unbound by the graph. Given two nodes $v$ and $u$ in the plane. Then $\overline{vu}$ will indicate that there is an edge between node $v$ and $u$, and given an edge $x$ in the graph, then $|x|$ is the length of $x$ in the Euclidean space.

The object of routing in any network is to send a \emph{message} -- which contains both the data to be delivered, but also the meta-data needed for routing, like last visited node -- from its origin, in this thesis defined as the \emph{source}, to some destination node, the \emph{sink}. The ordered set of nodes that the messages passes while it is being routed from the source to the sink will in this thesis be defined as the \emph{path}. In most routing algorithms, the goal is to find the \emph{shortest path} from the source to the sink, where the shortest path can either be the path that contains the fewest nodes/edges (sometimes referred to as \emph{hops}), or the path that has the least weight.

In some cases a node will have to route to  multiple sources and sinks, but in this thesis I will only focus on cases where there is exactly one of each. I will now introduce the terms I will use to describe the network.

\tikfig{node_different_radius}{Two nodes with different broadcast ranges. Node u is not adjacent to node v, but node v is adjacent to node u -- thus only one-way communication from u to v can take place.}

Given two nodes $v$ and $u$ in the graph, I will say that $u$ is \emph{adjacent} to $v$, iff $v$ can send a message to $u$ directly (i.e. $u$ lies within the range of $v$ transmitter). However, just because $u$ is adjacent to $v$ does not mean that the two nodes can communicate with each other\footnote{Two-way communication is important, since in most routing protocols, $v$ would need a confirmation that $u$ did indeed receive its message, this might be impossible with one-way communication.} since we do not know whether or not $v$'s transmitter is more powerful than $u$'s -- see figure~\ref{fig:node_different_radius}. 

One way to avoid this problem is to apply the \ac{uga}. The \ac{uga} states that all nodes has a perfect omni-directional transmission, that they all spend the same amount of power on the transmission, and that ``the energy dissipates as the square of distance'' \cite{practical}. For the sake of convenience I will employ the \ac{uga} in this thesis.\label{section:apply_uga}

More generally I will say that two nodes nodes $v$ and $u$ can only communicate if there exists a \emph{bidirectional link} between the nodes, and that a bidirectional link exist iff $v$ is adjacent to $u$ and $u$ is adjacent to $v$.

Now we can begin to define the \emph{connection graph}. Given a set of nodes $V$ the connection graph C is defined as $C = (V, E)$, where the set of edges $E$ contains an edge between node $v$ and $u$ iff there is a bidirectional link between $v$ and $u$ -- see figure~\ref{fig:connec}. The connection graph only tells us which nodes that are able to communicate with each other, not how the messages are going to be routed. Therefore I introduce the \emph{Routing Graph} R, where $R = (V, E^{\prime})$, where $E^{\prime} \subseteq E$. It is this graph $R$ that the message are routed through in order to find a path from the source to the sink.

Since we work under the \ac{uga}, and therefore want the path from the source to the sink with the fewest hops possible, it is useful to introduce the notion of a $k$-spanner. Given a graph $G = (V, E)$, we say that the sub-graph $G^{\prime} = (V, E^{\prime})$ is a $k$-spanner, where $k$ is a constant, if for all pairs of points in $G$, the same pair in $G^{\prime}$ is at most $k$ times as far apart as in $G$. This is an obviously desirably property as it gives us an upper bound on how good $G^{\prime}$ is compared to $G$. I will deal further with spanners in section~\ref{section:spanners} on p.~\pageref{section:spanners}.

\tikfig{connec-route}{Figure~\ref{fig:nodes} shows a set of nodes. The circle indicates the range of node a's transmission -- all the other nodes has the same range, but have been omitted for clarity's sake. Figure~\ref{fig:connec} shows us the connection graph given the nodes and transmission radius indicated in figure~\ref{fig:nodes}. Figure~\ref{fig:routing_not_planar} and \ref{fig:routing_planar} are examples of routing graph, a subset of the connection graph. Since figure~\ref{fig:routing_planar} is planar, it would be preferred over figure~\ref{fig:routing_not_planar}.}

\subsection{Topology}
\label{section:topology}
\tikfig{graphTop}{An example of the difference between Geometrical and Topological movement. For clarity I have opted not to include node c's transmission radius. In \ref{fig:graphTop2} node b has moved around inside the intersection of node a and node c's transmission radius, but since there has been no change in which nodes node b is adjacent to, b is still a neighbour to both nodes. In \ref{fig:graphTop3} node b has moved outside of node a's transmission radius, but is still inside node c's radius, changing the topology of the graph.}

Since we are dealing with mobile units, the topology of the graph may change as the nodes move around, which means that there must be a mechanism that updates the topology. Since each node has an area it can monitor, its \emph{transmission area}, we will have to be careful to distinguish between \emph{Geometrical movement}, where the node moves, but the topology of the network stays the same, and \emph{Topological movement}, where the movement of the node causes a change in the topology. I have illustrated the difference in figure~\ref{fig:graphTop}.

In the rest of the thesis I will assume we are working in an idealised world under the \ac{uga}, where the transmission radius is a perfect unit disc, and that the nodes can compensate for the transmission energy dissipation, as long as they are within the disc. I make this simplification since the goal of this thesis is not to create a real-world system, but to explore the requirements of the \ac{manet} in general, and the graph and routing protocols in particular.
