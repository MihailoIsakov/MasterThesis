\section{Routing graph}
\label{section:routing_graph}

Having defined the problem of routing in Section~\ref{section:fundamental_graph_routing}, I will in this section describe some of the different kinds of Routing graphs that can be used for \acp{manet}, and will describe the one I have used for this thesis. Furthermore I will give an introduction to single-, two-, and multi-level routing node organisation. 

In the literature I have read, I have not been able to find a consensus on how the optimal graph for \ac{manet} should be structured. However, it is clear from the literature that the structure of the graph heavily influences the routing algorithm, making them heavily dependant on each other. 

\subsection{Single-level graphs}
\label{single_level}
The first important question should be whether the graph should be planar or not. A planar graph is defined as a graph that can be embedded in the plane -- where one of the properties of a planar embedding is that no edge will cross another edge.
Curiously I have not been able to find any literature on the topic of \ac{anet}, which discuss whether or not the graph should be planar, as it seems a given that the graph must be planar, and then tries to find the optimal planar graph. I will therefore in this section debate both cases, hopefully giving the reader better insight why, or why not to use planar graphs.

From figure~\ref{fig:routing_not_planar} and figure~\ref{fig:routing_planar} it is clear that in non-planar graph, each node can communicate directly with more nodes than compared to the planar graph. Since non-planar graphs are allowed to have edges that overlap, while planar graphs are not, it is clear that given a set of nodes, one would always be able to construct a non-planar graph whose nodes would be better (or at least just as well) connected than the nodes in any planar graph. The question then becomes whether the non-planar graph is well suited for a \ac{manet}, or a planar graph would be better. The advantage of having a non-planar graph is that there is a faster route from the source to the sink (since more edges are allowed), but its problematic from a theoretical and practical perspective.

From a theoretical perspective it is clear that we know less about a non-planar graph than that of a planar one. For instance we have Euler's formula, which states that if we let $v$ be the number of vertices's, $f$ the number of faces and $e$ the number of edges then $v - e + f = 2$. Thus we can begin to reason about the ratio of the graph if we are given one of $e$, $f$ or $v$. This also becomes a problem with routing, as it likewise becomes harder to reason about routing with non-planar graphs\hide{, and most traditional routing algorithms assume to work on planar graphs\todo{do traditional routing algorithms assume that the graph is planar?}}. It is also important to remember that a routing algorithm working on a non-planar graph will have to work on more edges, which will slow it down.

From a practical perspective working with non-planar graphs will mean that each node will have to keep track of more neighbours (since there is no theoretical upper bound on the nodes in the vicinity, though realistically I would say that more than 50 neighbours are unlikely, even for crowded areas). In dense networks the extra neighbours added by having a non-planar graph can be significant --- see figure \todo{make crowded street example} for an illustration of this. The number of neighbours for each node also becomes a problem with routing schemes that explicitly sends information to a single neighbour instead of flooding the entire network, as more nodes have to be inspected, than if we had a planar graph. While more energy has to be used to keep track of neighbours, this does not mean a greater transmission overhead, as we per our \ac{uga} assumption are transmitting our data omni-directional, and not directly to each neighbour in a turn by turn fashion.

From this I conclude that while a non-planar graph can be used as an alternative to planar graphs, I do not feel that the loss of knowledge and extra bookkeeping is worth it. I will therefore in this thesis only pursue planar graphs. However, in section
 
Having chosen to keep our graphs planar we must take care that the routing graphs will always be planar.

I have found the following general methods for making the planar graphs:
\begin{description}
\item[The \ac{gabe}:] The \ac{gabe} is a subset of the Delaunay triangulation, where two nodes $v$ and $u$ only share an edge iff the circle that has $\overline{vu}$ as its diameter does not contain a node $q$. In practice a good way of computing the \gabe{gabe} is to use the Delaunay triangulation to limit the number of node pairs that have to be checked.
\begin{algorithmic}[H]
  \caption{GabrielGraph}
  \SetKwData{edge}{edge}
  \SetKwData{pOne}{p1}\SetKwData{pTwo}{p2}
  \SetKwData{mid}{midPoint0}
  \setKwInOut{Input}{input}
  \setKwInOut{Output}{output}
  \dontprintsemicolon
  \Input{A set of points P}
  \Output{The Gabriel Graph based on P}
  delaunay $\gets$ MakeDelaunay(P) \;
  \ForEach{\edge $\in Edges(delaunay)$}{
    \pOne $\gets FirstPoint(\edge)$ \;
    \pTwo $\gets SecondPoint(\edge)$ \;
    \mid $\gets FindMidPoint(\pOne, \pTwo)$ \;
 

    }
\STATE 
\end{algorithmic} 

\cite{gpsr, gopher}. 
\item[The \ac{rng}:] Two nodes $v$ and $u$ share an edge if there is no node $q$ such that $|\overline{vq}|, |\overline{uq}| \leq |\overline{vu}|$ (i.e. $q$ is closer to both $v$ and $u$ than $v$ and $u$ are to each other). \cite{gpsr, RNG}
\item[The \ac{cldp}:] \ac{cldp} works by having each node uses \emph{the right-hand-rule}\label{right-hand rule}\footnote{The right-hand-rule states that for routing graphs, if you chose to go to the right each time you have a choice, you will eventually end up where you started.} and checks whether or not any of its links crosses/is crossed by another edge. In other words, \ac{cldp} first builds up a graph that may be non-planar, and then in a distributed manner finds and removes edges that makes the graph non-planar without disconnecting parts of graph (as the paper accuses both \ac{gabe} and \ac{rng} of). \ac{cldp} is not dependent on the Unit graph assumption \cite{practical}. 
\end{description}

Both the \ac{gabe} and the \ac{rng} are based on the \ac{uga}, and can therefore have problems in real world situations, accidentally removing edges that should otherwise have been there\cite{practical}.


The \ac{gabe} is a sub-graph of the Delaunay triangulation, and \ac{rng} is a subset of \ac{gabe} \todo{find and cite}, and as such they are both well described and understood graphs many of the same properties as the Delaunay triangulation. \ac{cldp} on the other hand is not a graph, but rather an algorithm that takes an already existing graph and makes sure its planar, which, according to \cite{practical} is less likely to make the graph disconnected than \ac{gabe} and \ac{rng}. It is therefore harder to reason about most probable maximum number of neighbours for \ac{cldp}. 

\tikfig{graph_comparison}{\ref{fig:rng1} and \ref{fig:rng2} are examples of the \ac{rng}, while \ref{fig:gabe1} and \ref{fig:gabe2} are examples of the \ac{gabe}. This figure is inspired by figure~7 in \cite{GeoSpanners}.}

Delaunay graphs have been found to have an expected maximum degree of $\theta(\log n / \log \log n)$ \cite{delExpected}, and furthermore that that \ac{gabe} has a maximum degree of $\theta(\log n / \log \log n)$, or, as the authors put it, there exists a constant $a$ and constant $b$ such that 
$$
\lim_{n \rightarrow \infty} Pr\left({\Delta \in \left[\frac{a\log n}{\log \log n}, \frac{b\log n}{\log \log n}\right]}\right) = 1
$$
and for \ac{gabe} $a = 1/12$ and $b = 1$, for uniform distributions \cite{GGExpected}.

Since the \ac{rng} is a subset of the \ac{gabe}, it can at most have as many neighbours. A comparison between the \ac{gabe} and the \ac{rng} is illustrated in figure~\ref{fig:graph_comparison} and figure~\ref{fig:gg-rng-example-big}. For more on the difference between the \ac{gabe} and the \ac{rng} in practise see section~\ref{section:spanners} on p.~\pageref{section:spanners}.

%\tikfig{gg-rng-example-big}{In the following examples all the nodes have the same transmission radius. In figure~\ref{fig:norm_graph} we see the non-planar graph. In figure~\ref{fig:gg_graph} we see the \ac{gabe} for the same set of points, clearly being planar and with far less edges, and in figure~\ref{fig:rng_graph} we see the \ac{rng} graph, which on closer inspection turns out to be a subset of the \ref{fig:gg_graph}. For comparison we have the Minimum Spanning tree in figure~\ref{fig:mst}}
\todo{unhide when we have to print}

It is not immediately apparent how one would compare \ac{gabe} and \ac{rng} to \ac{cldp} on a theoretical level, since \ac{cldp} is effected on the run-time behaviour of the transmission between the nodes. While I could implement \ac{cldp} in practise and compare the results to the non-planar graphs and \ac{gabe} and \ac{rng} I have chosen not to do this, as I feel it would be to much work when figuring in the relative obscureness of \ac{cldp} compared to \ac{gabe} and the \ac{rng}.  

Neither the \ac{manet} or the \ac{anet} can be said to be entirely static. Even in an \ac{anet}, where we assume the nodes remain stationary, every single node will eventually run out of energy or fail. The routing graph must therefore always have a maintenance system that can deal with changes to the underlying graph. 

To complicate matter further different nodes can in practise have different communications radius as exemplified in figure~\ref{fig:node_different_radius}, which shows us that a node needs to confirm that it indeed has a bidirectional link to each of its adjacent nodes, and cannot in practise take it for granted.

\tikfig{different_radius}{A node with different broadcast ranges. The first solid circle is the default broadcast range of node a, which does not reach node b. The second circle is the maximum range that a can broadcast, which does reach node b.}

Likewise it is important to remember that a node in practise might be able to adjust its communications range (at a higher energy cost). See figure~\ref{fig:different_radius}

\subsection{Two-Level graphs}
\tikfig{gateway-node}{Node $s$ and $t$ are the source and the sink respectively, the rectangular nodes with $CH$ are the \acp{ch} and the diamond nodes are the gateway node, and the entirely white or black nodes are the regular nodes that have elected the right- and leftmost \ac{ch} respectively. From the figure we can see that the message travels from the source to the header, between the gateways and then to the second  and then lastly to the sink, as indicated by the grey background. It is also clear that it would have been faster to go from the gateway and then to the sink, but since the gateway in this model does not know the location of the different nodes, the message will first have to be routed to the rightmost node.}

\label{section:cluster_methods}
Node Clustering is a hierarchical alternative to the flat graph. Instead of having every node being the same, a number of nodes are made into \acp{ch} such that all nodes in the graph has 1 cluster head that they communicate with. Most of the time the nodes that belong to a certain \ac{ch} is its one-hop neighbours, but there are also schemes that allows nodes to have a clustered that is farther away, in order to reduce the number of \acp{ch}. With \acp{ch} a layer specialisation and hierarchy is added to the network, which reduces the complexity of an otherwise entirely flat network, while still retaining some decentralisation.

In order to improve the routing, \acp{ch} are almost always accompanied by \emph{gateway nodes}. Gateway nodes are nodes that connect two clusters --- see figure~\ref{fig:gateway-node} for an illustration. \cite{spanners} defines a gateway node as follows:
Let C(p) be the set of nodes that has the cluster p as its \ac{ch} (this includes p). Then for a pair of \acp{ch} $c_1, c_2$ if there exists a pair of nodes $p_1 \in C(c_1), p_2 \in C(c_2)$ such that $p_1$ and $p_2$ can communicate with each other in the graph, then $p_1$ and $p_2$ are gateway nodes (and in particular they are the only gateway nodes between $c_1$ and $c_2$.  

Some of the practical advantages are:
\begin{itemize}
\item The \ac{ch} will cut down on the number of necessary transmissions needed to check whether nodes are alive or not.
\item \acp{ch} makes routing simpler, since all transmissions will have to go though the node gateways or up through the chain of \acp{ch}. Thus only the \acp{ch} and gateway nodes will need to update their location database. Messages will still have to be sent through regular nodes (as illustrated in figure~\ref{fig:gateway-node}), but instead of having to route directly from the source to the sink, the message will alternating be routed through \acp{ch} and gateways, until it reaches the sink.
\end{itemize}

Using \acp{ch} we divide the communication between nodes into 3 stages:
\begin{enumerate}
\item From the source to the local \ac{ch} 
\item From the local \ac{ch} to the \ac{ch} closest to the sink
\item From the \ac{ch} closets to the sink to the sink itself\footnote{If either \ac{ch} is the sink or the source, and the first or third part is of course eliminated}.
\end{enumerate}\todo{refernce the figure of this}
 \acp{ch} thus decreases the number of nodes that needs to keep track of the other nodes, and the gateway nodes divides the routing into multiple smaller trips.  

However, since we are introducing a layer of centralisation, we must ensure that we can detect and replace a \ac{ch} should it fail (and to elect it in the first place). An often used system is a voting system, or an selection algorithm (based hash value of the node's id, its MAC value etc.).

One way of exploiting routing between \acp{ch} is to use the \ac{rdg}:
\begin{description}
\item[The \ac{rdg}:] The \ac{rdg} is a graph scheme that employs \acp{ch} based planar graph solution. After having found the set $C$ of \acp{ch}, each $ch \in C$ calculates its local Delaunay edges on the set $C$, and then confers with its one-hop \ac{ch} neighbours to check which Delaunay edges are valid. Besides being planar and computable in a distributed manner, the \ac{rdg} is also a spanner, which means that using it will only make the route a constant times longer than the optimal route \cite{GeoSpanners}.
\end{description}

\subsection{Multiple-level graphs}
A logical extension to the Two-level graph is the multiple-level graph. In the multiple-level graph the $[n]$th level contains the location of the nodes on the $[n-1]$th level, for $n \geq 3$. The selection for each level can be done similar to when dealing with second-level graphs, with redefined neighbours    
 \todo{find article and cite}

\subsection{Spanners}
\label{section:spanners}
As mentioned in \ref{section:graph_concepts} it would be very nice if we could get a guarantee on how effective a given graph building algorithm would be. However, before looking further into this we must first decide what a spanner means for a \ac{manet}.

If we were given a set of nodes in the plane, and told to create the graph that would leave us with the least distance under the euclidean metric, between any pair of points, then the best choice would clearly be the complete graph. This would also be the shortest distance if we use the number of hops as the primary metric. Using the complete graph is of course inefficient (and impossible if the graph has more than 4 nodes and has to be planar), and multiple algorithms and graph types have been invented to find an acceptable trade-off between the number of edges, extra nodes (such as the ones found in Steiner-trees), and the distance or hops between any pair of nodes.

When trying to apply spanners to \acp{manet} we have to remember that each node has a limited broadcast range, creating an upper bound on the length of an edge. Depending on the scheme and the hardware of the node, each edge may cost the same, no matter its length (since messages are not sent as directed beams, but rather transmitted omni-directionally), or there may be an option to increase the radius of the signal for an additional cost. In the following I will assume the fixed cost-model, as it is far more likely to be used than variable-cost model. Because of these differences in spanners for the \acp{manet}, I will therefore use the term \emph{limited-range spanner} to distinguish them from the concept as it is normally used. 

Since I have been unable to find empirical data detailing the effectiveness of either the \acp{gabe} or \acp{rng} compared the to the non-planar graph, I have decided to empirically test the three different graph types. For more information about the setup of this test see section \ref{section:test_desc_spanners} and section \ref{section:test_results_spanners} for the results.
