\section{Routing graph}
\label{section:routing_graph}

Having defined the problem of routing in Section~\ref{section:fundamental_graph_routing}, I will in this section describe some of the different kinds of Routing graphs that can be used for \acp{manet}, and will describe the one I have used for this thesis. Furthermore I will give an introduction to single-, two-, and multi-level routing node organisation. 

In the literature I have read, I have not been able to find a consensus on how the optimal graph for \ac{manet} should be structured. However, it is clear from the literature that the structure of the graph heavily influences the routing algorithm, making them heavily dependant on each other. 

\subsection{Single-level graphs}
\label{single_level}
The first important question should be whether the graph should be planar or not. A planar graph is defined as a graph that can be embedded in the plane -- where one of the properties of a planar embedding is that no edge will cross another edge.

Curiously I have not been able to find any literature on whether or not a graphs for \acp{anet} or \acp{manet} should be planar. Rather it seems like it is a given that the graph must be planar in order to make the \ac{anet}/\ac{manet}, and then tries to find the optimal planar graph. I will therefore in this section debate both cases, hopefully giving the reader better insight why, or why not to use planar graphs.

From figure~\ref{fig:routing_not_planar} and figure~\ref{fig:routing_planar} it is clear that in the non-planar graph, each node can communicate directly with more nodes than compared to the planar graph. Since graphs that are allowed to be non-planar are allowed to have edges that overlap, while planar graphs do not, it is clear that given a set of nodes, one would always be able to construct a graph without the restriction of being planar, whose nodes could be better or at just as well connected as any planar graph. The question then becomes whether a graph without the planar restriction would be well suited for a \ac{manet}, or a planar graph would be better. Since there are many ways to ways to define a graph that is not restricted to being planar and since we are dealing with \acp{manet}, where each node only has a finite transmission radius, I will introduce the \ac{ucg}, which is the only variant of graph without any planar restrictions I will discuss in this thesis. The \ac{ucg} is the graph $G$ where for all points $p$ and $q$ in $G$, there is an edge between $p$ and $q$ iff the distance between $p$ and $q$ is less than a radius $r$. An example of the \ac{ucg} can be found in figure~\ref{fig:norm_graph}. 

The advantage of working with a \ac{ucg} is that there likely would be a faster route from the source to the sink (since more edges would be allowed), but its problematic from a theoretical and practical perspective.

From a theoretical perspective it is clear that we know less about a graphs that might not be planar, than of a graphs that are sure to be planar. For instance we have Euler's formula, which states that if we let $v$ be the number of vertices's, $f$ the number of faces and $e$ the number of edges then $v - e + f = 2$. Thus we can begin to reason about the ratio of the graph if we are given one of $e$, $f$ or $v$. This also becomes a problem with routing, as we now know less about the properties of any nodes neighbour, and will also make it impossible to use routing algorithms, such as GPSR\footnote{See section~\ref{section:gpsr} for more on GPSR.} requires the graph to be planar. It is also important to remember that a routing algorithm working on a non-planar graph will have more neighbours than a planar graph, which would require more memory and make routing descisions take longer.

From a practical perspective working with \ac{ucg} will mean that each node will have to keep track of more neighbours (since there is no theoretical upper bound on the nodes in the vicinity). In dense networks the extra neighbours an \ac{ucg} will have compared to a planar graph can be significant --- see figure~\ref{fig:gg-rng-example-big} on p. \pageref{fig:gg-rng-example-big} for an illustration. The number of neighbours for each node also becomes a problem with routing schemes that explicitly sends information to a single neighbour instead of flooding the entire network, as more nodes have to be inspected, than if we had a planar graph. While more energy has to be used to keep track of neighbours, this does not cause a greater transmission overhead, as we per our \ac{uga} assumption are transmitting our data omni-directional, and not directly to each neighbour in a turn-by-turn fashion.
 
From this I conclude that while a \ac{ucg} can be used as an alternative to planar graphs, they are not desriable from a theoretic perspective, considering the loss of knowledge about the graph and the extra bookkeeping. However, \ac{ucg} is an interesting baseline to compare other planar graphs with in practice, which I have done in section~\ref{section:spanners}.
 
I have found the following popular methods for making planar graphs:
\begin{description}
\item[The \ac{gabe}:] The \ac{gabe} is a subset of the Delaunay triangulation, where two nodes $v$ and $u$ only share an edge iff the circle that has $\overline{vu}$ as its diameter does not contain a node $q$. In practise a good way of computing the \ac{gabe} is to first do a Delaunay triangulation to limit the number of node pairs that have to be checked.

\cite{gpsr, gopher}. 
\item[The \ac{rng}:] Two nodes $v$ and $u$ share an edge if there is no node $q$ such that $|\overline{vq}|, |\overline{uq}| \leq |\overline{vu}|$ (i.e. $q$ is closer to both $v$ and $u$ than $v$ and $u$ are to each other). \cite{gpsr, RNG}
\end{description}

\label{del_gabe_rng_neigh}It should be noted that the \ac{gabe} is a sub-graph of the Delaunay triangulation\cite{GGExpected}, and that \ac{rng} is a subset of \ac{gabe} \cite{GGExpected}, which makes it easier for us to reason and compare the graphs.

\tikfig{graph_comparison}{\ref{fig:rng1} and \ref{fig:rng2} are examples of the \ac{rng}, while \ref{fig:gabe1} and \ref{fig:gabe2} are examples of the \ac{gabe}. This figure is inspired by figure~7 in \cite{GeoSpanners}.}

In the Delaunay graph there is an edge $\overline{vu}$ between two points $v$ and $u$ if and only if there is a closed disc $C$ with $v$ and $u$ on the boundary and no point on the inside of $C$ \cite{CompuGeo}. From this it is clear that in theory there is no upper limit on how many neighbours a point can have in the Delaunay graph, since an infinite number of nodes could be placed on the border of $C$. However, this is clearly a degenerate case, and we must therefore instead try to reason about the expected maximum degree, and doubly true when we are talking about the \ac{gabe} and the \ac{rng}.

Delaunay graphs have been found to have an expected maximum degree of $\Theta(\log n / \log \log n)$ \cite{delExpected} --- where $\log$ is the natural logarithm, and furthermore that that \ac{gabe} has an expected maximum degree of $\Theta(\log n / \log \log n)$ \cite{GGExpected}, or, as the authors in \cite{GGExpected} puts it, there exists a constant $a$ and $b$ such that if $\Delta$ is the maximum degree of the \ac{gabe} then 
$$
\lim_{n \rightarrow \infty} Pr\left({\Delta \in \left[\frac{a\log n}{\log \log n}, \frac{b\log n}{\log \log n}\right]}\right) = 1
$$
and for \ac{gabe} $a = 1/12$ and $b = 1$, for uniform distributions \cite{GGExpected}.

Since the \ac{rng} is a subset of the \ac{gabe}, it can at most have as many neighbours. A comparison between the \ac{gabe} and the \ac{rng} is illustrated in figure~\ref{fig:graph_comparison} and figure~\ref{fig:gg-rng-example-big}. For more on the difference between the \ac{gabe} and the \ac{rng} in practise see section~\ref{section:spanners} on p.~\pageref{section:spanners}.

\tikfig{gg-rng-example-big}{In the following examples all the nodes have the same transmission radius. In figure~\ref{fig:norm_graph} we see the \ac{ucg}. In figure~\ref{fig:gg_graph} we see the \ac{gabe} for the same set of points, clearly being planar and with far less edges, and in figure~\ref{fig:rng_graph} we see the \ac{rng} graph, which on closer inspection turns out to be a subset of the \ref{fig:gg_graph}. For comparison we have the Minimum Spanning tree in figure~\ref{fig:mst}}

Neither the \ac{manet} or the \ac{anet} can be said to be entirely static. Even in an \ac{anet}, where we assume the nodes remain stationary, every single node will eventually run out of energy or fail. The routing graph must therefore always have a maintenance system that can deal with changes to the underlying graph. 

To complicate matter further different nodes can in practise have different communications radius as exemplified in figure~\ref{fig:node_different_radius}, which shows us that a node needs to confirm that it indeed has a bidirectional link to each of its adjacent nodes, and cannot in practise take it for granted.

\tikfig{different_radius}{A node with different broadcast ranges. The first solid circle is the default broadcast range of node $u$, which does not reach node $v$. The second circle is the maximum range that node $u$ can broadcast, which does reach node $v$.}

Likewise it is important to remember that a node in practise might be able to adjust its communications range (at a higher energy cost). See figure~\ref{fig:different_radius}

\subsubsection{The \ac{cldp}}
Having mentioned both the \ac{gabe} and the \ac{rng}, both of which uses the \ac{uga}, it would be amiss not to mention at least one variation that does not use the \ac{uga}. The only one I have been able to find is the \ac{cldp}.
 
The \ac{cldp} works by first building a graph that may be unit-complete, and then have each node in the network identify which edges overlaps, and then use a distributed algorithm to attempt to fix it:

After the initial graph is built, each $u$ in the graph transmits a message to each of its neighbours. Let us for the examples sake call one of its neighbours $v$, and the message it is sent $m_{\overline{uv}}$. The goal of $m_{\overline{uv}}$ is, as described above, to check if there is any edge that crosses $\overline{uv}$. $m_{\overline{uv}}$ will then traverse the graph using the \label{right-hand-rule} \emph{the right-hand-rule}\footnote{The right-hand-rule states that in routing for routing graphs, if you chose to go to the right each time you have a choice, you will eventually end up where you started.}, while recording the number of times it passes each edge.

Once $m_{\overline{uv}}$ arrives at a node $q$ with a rightmost neighbour $p$, it will check whether $\overline{qp}$ crosses $\overline{uv}$, if it does, then $\overline{qp}$ will be recorded as doing so. Once $m_{\overline{uv}}$ arrives back at the $u$, it goes through all the edges that crosses $\overline{uv}$, and tries to resolve the situation. The authors note that if the message has traversed an edge twice, then removing that edge could partition the graph. Therefore an edge that has been traversed twice is said to be \emph{non-removable}. If we let $e$ be the first crossed edge and $e^{\prime}$ be a line that crosses $e$, then \ac{cldp} uses the following cases to decide which edge to remove:
\begin{description}
\item{Case 1:} If both $e$ and $e^{\prime}$ ca be removed, then remove $e$.
\item{Case 2:} If $e$ can be removed, but $e^{\prime}$ cannot, then $e$ is removed.
\item{Case 3:} If $e$ cannot be removed, but $e^{\prime}$ can, then remove $e^{\prime}$.
\item{Case 4:} If neither edge can be removed, do nothing. 
\end{description}

The observant reader will note that the fourth case makes it possible for the graph generated by the \ac{cldp} to be unit-complete, which is the reason for its non-inclusion in the previous section. However, to make up for this deficiency the authors note that in real world situations where the \ac{uga} does not always hold (specifically nodes may have different transmission radii), application of both the \ac{gabe} and the \ac{rng} may partition the network \cite{practical}. 

Since the \ac{cldp} uses a distributed algorithm of far greater complexity than both the \ac{gabe} and the \ac{rng} (and that does not require any message passing) and that the graph produced by the \ac{cldp} may be unit-complete, the \ac{cldp} will not be analysed further in this thesis.

\subsection{Two-Level graphs}

\tikfig{gateway-node}{Node $s$ and $t$ are the source and the sink respectively, the rectangular nodes with $CH$ are the \acp{ch} and the diamond nodes are the gateway node, and the entirely white or black nodes are the regular nodes that have elected the right- and leftmost \ac{ch} respectively. From the figure we can see that the message travels from the source to the header, between the gateways and then to the second  and then lastly to the sink, as indicated by the grey background. It is also clear that it would have been faster to go from the gateway and then to the sink, but since the gateway in this model does not know the location of the different nodes, the message will first have to be routed to the rightmost node.}

\label{section:cluster_methods}
Node Clustering is a hierarchical alternative to the flat graph. Instead of having every node being the same, a number of nodes are made into \acp{ch} such that all nodes in the graph has 1 cluster head that they communicate with. Most of the time the nodes that belong to a certain \ac{ch} is its one-hop neighbours, but there are also schemes that allows nodes to have a clustered that is farther away, in order to reduce the number of \acp{ch}. With \acp{ch} a layer specialisation and hierarchy is added to the network, which reduces the complexity of an otherwise entirely flat network, while still retaining some decentralisation.

In order to improve the routing, \acp{ch} are almost always accompanied by \emph{gateway nodes}. Gateway nodes are nodes that connect two clusters --- see figure~\ref{fig:gateway-node} for an illustration. \cite{spanners} defines a gateway node as follows:
Let C(p) be the set of nodes that has the cluster p as its \ac{ch} (this includes p). Then for a pair of \acp{ch} $c_1, c_2$ if there exists a pair of nodes $p_1 \in C(c_1), p_2 \in C(c_2)$ such that $p_1$ and $p_2$ can communicate with each other in the graph, then $p_1$ and $p_2$ are gateway nodes (and in particular they are the only gateway nodes between $c_1$ and $c_2$.  

Some of the practical advantages are:
\begin{itemize}
\item The \ac{ch} will cut down on the number of necessary transmissions needed to check whether nodes are alive or not.
\item \acp{ch} makes routing simpler, since all transmissions will have to go though the node gateways or up through the chain of \acp{ch}. Thus only the \acp{ch} and gateway nodes will need to update their location database. Messages will still have to be sent through regular nodes (as illustrated in figure~\ref{fig:gateway-node}), but instead of having to route directly from the source to the sink, the message will alternating be routed through \acp{ch} and gateways, until it reaches the sink.
\end{itemize}

Using \acp{ch} we divide the communication between nodes into 3 stages:
\begin{enumerate}
\item From the source to the local \ac{ch} 
\item From the local \ac{ch} to the \ac{ch} closest to the sink
\item From the \ac{ch} closets to the sink to the sink itself\footnote{If either \ac{ch} is the sink or the source, and the first or third part is of course eliminated.}.
\end{enumerate}
See figure~\ref{fig:gateway-node} for an illustration of this.

 \acp{ch} thus decreases the number of nodes that needs to keep track of the other nodes, and the gateway nodes divides the routing into multiple smaller trips.  

However, since we are introducing a layer of centralisation, we must ensure that we can detect and replace a \ac{ch} should it fail (and to elect it in the first place). An often used system is a voting system, or an selection algorithm (based hash value of the node's id, its MAC value etc.).

One way of exploiting routing between \acp{ch} is to use the \ac{rdg}:
\begin{description}
\item[The \ac{rdg}:] The \ac{rdg} is a graph scheme that employs \acp{ch} based planar graph solution. After having found the set $C$ of \acp{ch}, each $ch \in C$ calculates its local Delaunay edges on the set $C$, and then confers with its one-hop \ac{ch} neighbours to check which Delaunay edges are valid. Besides being planar and computable in a distributed manner, the \ac{rdg} is also a spanner, which means that using it will only make the route a constant times longer than the optimal route \cite{GeoSpanners}.
\end{description}

\hide{
\subsection{Multiple-level graphs}
A logical extension to the Two-level graph is the multiple-level graph. In the multiple-level graph the $[n]$th level contains the location of the nodes on the $[n-1]$th level, for $n \geq 3$. The selection for each level can be done similar to when dealing with second-level graphs, with redefined neighbours    
 \todo{find article and cite}
}

\subsection{Spanners}
\label{section:spanners}
As mentioned in \ref{section:graph_concepts} it would be very nice if we could get a guarantee on how effective a given graph building algorithm would be. However, before looking further into this we must first decide what a spanner means for a \ac{manet}.

If we were given a set of nodes in the plane, and told to create the graph that would leave us with the least distance under the euclidean metric, between any pair of points, then the best choice would clearly be the complete graph. This would also be the shortest distance if we use the number of hops as the primary metric. Using the complete graph is of course inefficient (and impossible if the graph has more than 4 nodes and has to be planar), and multiple algorithms and graph types have been invented to find an acceptable trade-off between the number of edges, extra nodes (such as the ones found in Steiner-trees), and the distance or hops between any pair of nodes.

When trying to apply spanners to \acp{manet} we have to remember that each node has a limited broadcast range, creating an upper bound on the length of an edge. Depending on the scheme and the hardware of the node, each edge may cost the same, no matter its length (since messages are not sent as directed beams, but rather transmitted omni-directionally), or there may be an option to increase the radius of the signal for an additional cost. In the following I will assume the fixed cost-model, as it is far more likely to be used than variable-cost model. Because of these differences in spanners for the \acp{manet}, I will therefore use the term \emph{limited-range spanner} to distinguish them from the concept as it is normally used. 

Since I have been unable to find empirical data detailing the effectiveness of either the \acp{gabe} or \acp{rng} compared the to the \ac{ucg}, I have decided to empirically test the three different graph types. I will do so for a single-level graph and will therefore not be including the \ac{rdg} --- interested parties should consult \cite{GeoSpanners} for a comparison between the \ac{gabe} and the \ac{rdg}. Also I will not test the effectiveness of the \ac{cldp}, as it would require too much time to implement, and that its missing the theoretical background that makes both the \ac{gabe} and the \ac{rng} interesting.

For more information about the setup of this test see section \ref{section:test_desc_spanners} and section \ref{section:test_results_spanners} for the results.
