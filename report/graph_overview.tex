\documentclass[letter, 12pt, english, draft]{article}
\usepackage{mikkel}
\usepackage[ruled,vlined]{algorithm2e}
\newcommand{\halfblankline}{\quad\vspace{-0.5\baselineskip}\pagebreak[3]}
\newcommand{\propSol}[2]{\texttt{Problem:}\\
#1\\
\texttt{Solution:}\\
#2
\halfblankline
}
\title{Energy Affluent Mobile Ad-Hoc Networks}
\author{Mikkel Kjær Jensen}
\date{\today}
\begin{document}

\pagestyle{fancy}

\section{Routing graph}

In the case of mobile ad-hoc networks, the routing graph does not have a singular structure. Since nodes will be able to broadcast and receive signals to/from nodes in a radius (at least under optimal conditions) around them, the most routing graph with the maximum number of edges will not be planar. Furthermore, while most routing algorithms assume so, the edges do not have to be bidirectional. One could easily imagine a scenario where heterogeneous ad-hoc nodes had different broadcast radii, leading to cases where certain nodes could receive other nodes broadcast through 1 jump, but would have to route through several nodes to get the message back.

The preferred structure of the underlying graph is that of planar graph. This makes sense since many traditionally popular routing algorithms makes this assumption (and simplifies the situation immensely). There are several methods for turning the underlying structure into a planar graph (i.e. making sure that for any given node, the only neighbours they have are the ones that do not violate the planar nature of the graph, were it to be represented as a graph), and keeping it so for the lifetime of the network.

Since we are dealing with mobile units, the graph would be changing as the nodes moved around, which means that we would need a solution that could run continuously, making the necessary changes to the current structure, and not having to start over from scratch each time. Since we are dealing transmitting nodes, it is important to note that in practise changes is when a node leaves or enters another nodes radius (it does not matter which of them moves). While environmental factors will of course dampen signals, we can in most cases assume that the radio signal will behave more omni-directional than directional. Assuming we are sure that the desired node is in its transmit radius, a node can simply transmit its message (with some kind of recipient id --- unless it wants to flood the network), and hope that it avoids message collision.   

In practice no graph can be said to be entirely static. Even in cases where the nodes are completely staionary, all nodes in the ad-hoc network will eventually run out of energy or fail. The routing graph must therefore be amendable, even if the nodes are not moving. 

\section{Routing Algorithms}

The biggest conceptual difference for routing/shortest path algorithms in a mobile ad-hoc network compared to main-stray routing/shortest path algorithms, is the loss of global overview that these algorithms employ --- such as relaxing a specific nodes' edges at a specific time. There exist distributed version of some of the ``classic'' routing algorithms, but their usefulness may be limited in a mobile network, since they will have to be recomputed either each time a node wants to transmit a message, or each time the underlying graph changes.

Most of the following algorithms will mostly be based around the idea of \emph{local routing algorithms}. Paraphrasing \cite{compass}, a local routing algorithm is an algorithm that for fills the following criteria:
\begin{enumerate}
\item Each message will have a limited memory to store information about a constant number of vertices's, but it have full access to the coordinates of the starting position and the destination. At no point does the message have full knowledge of the entire graph.
\item Let $s$ be the source of the message, $t$ the sink, and $v$ be a node in the graph that can be reached from $s$. Arriving at $v$, the message can then use the information $v$ has stored about its neighbours, as well as its logical edges to the neighbours. Based on this information, one of the neighbours of $v$ is chosen as the next receiver of the message, unless $v = t$, in which case the message will not be propagated.
\item A message will leave a node in the same state as it was when it arrived. In this sense nodes are considered stateless.\end{enumerate}

Some routing algorithms, such as \cite{speed} violates the third property, writing information to the node in order to have congestion avoidance. Other algorithms uses the messages to have the nodes refresh their knowledge about the active nodes, and message ids of which messages that safely can be discarded. 

Routing algorithms suited for mobile ad-hoc networks can be sorted into the following categories:
\begin{description}
\item[Flooding:] The simplest of the routing algorithms. The entire networks is flooded with the message, ensuring that the message eventually reaches the sink. In order to ensure the network is not flooded with the same message forever, each node will need a mechanism that will allow it to record which messages they have already received, so that they can discard it if they ever receives it again. This technique is guarantied to work, but uses a lot of bandwidth.

\item[Directed Diffusion:] Directed Diffusion is an algorithm that can be described as a kind of directed flooding. Imagine a line-segment from the source to the sink, now sweep a disc (with a small radius) over the line-segment. The created figure by the sweep (which should resemble a cigar tube projected down onto the plane) is the idealised area the flooding will appear inside. This way the message will have a better chance of avoiding local minima, and will find its way to the sink. More advanced versions can then establish a permanent message route. In the case where both nodes that they must find each other, a bi-directional Directed Diffusion can take place, where each node tries to find each other, until both diffusion reach each other, and a path can be established.

\item[Greedy routing]: Greedy routing is when a message is transmitted, the next node will always be the node of the neighbours that is closest to the sink. If the message ever arrives at a node where all neighbours are farther away from the current node -- a local minimum -- then no action is specified, and message will not be delivered. The greedy algorithm thus offers no guarantee in delivering the message, but in practise it improves as the density of the network increases.

\item[Geometrical routing:] Geometrical routing uses the location of the current node and the location of the sink. There exist many different kinds of routing algorithms, but common for them is that unlike the greedy routing technique, they will transmit the message to a node that is not necessarily the closer node. This ensures that the message will eventually reach the sink.
 
\item[Shortest path:] The shortest path techinquie is to use distributed versions of the shortest path algorithms in order to find the shortest path from either one node to another, or from all nodes to all other nodes. As one might imagine, this approch has problems in mobile networks, where the structure of the graph can be in constant flux. Even in static networks the algorithms incur memory problems, since in the all-to-all solution, since for any node with a message for any node in the network must know which neighbour it must send it to get the most efficient route.  
\end{description}

Different routing algorithms make different assumptions about the underlying nature of the mobility of the graph. There are several routing algorithms \cite{two-tier}\cite{adaptive} that assumes a number of permanently stationary nodes or sensors as a part of the infrastructure, while still having a number of mobile nodes that needs to receive information. 

In practical situations the system will have to deal with all the problems inherent in wireless communication, such as the hidden terminal problem and message collision. To avoid these, having a lower communications layer that took care of this would be preferable to implementing it in the algorithm itself (though care should be taken to ensure that the algorithm is build to withstand these conditions).

\section{Problems we have with movement}

Movement presents us with the following problems:

\begin{itemize}
\item Changes underlying graph
\item Invalidates location information other nodes about about the node
\item If the node is part of path from \emph{a} to \emph{b} (and the node is neither), which is stored and which is believed to be used again, then that path may fail, and a longer path may have to be taken. In degenerate cases this may lead to situations where the new optimal path will lead away from the absolute coordinates of the end node.
\item The moving node may partition a network into 2 separate networks, if the node was the only node the 2 networks could communicate through
\item The moving node may create a link between 2 previously separate networks
\item The node may move into a position that creates a better path than the one before it moved - however, the surrounding nodes needs to be aware of it
\end{itemize}

The movement creates the need for the following solutions:
\begin{itemize}
\item There needs to be a way to restructure the underlying graph so that it reflects the movement of the node \cite{practical}
\item The location of the moving node needs to be updated. This does not necessary need to be done to all the nodes in the network, nor all the nodes that knew it before. There are many schemes for this, such as GLS \cite{scaleLocation} which partitions the world into squares, and then partitions these squares into other squares, resulting in several layers of squares. In each of these squares, there is a single node that knows the location of the node (and a scheme to find it).
\item Depending on implementation there may be no problem (there is a time-out system that detects that the message did not arrive, the node right before the missing node reports back that it could not deliver the message etc.). Finding a new path should not be a real problem unless the algorithm employed is a greedy algorithm (i.e. an algorithm that will always choose to go to the node that brings it closest to the absolute coordinates of the goal node).
\item If the network is partitioned, then the best course of action is to let the two networks figure out that the other part isn't available. Some kind of timeout mechanism in the location mechanism would seem like the most straightforward path
\item If connection is established between network \emph{a} and network \emph{b}, there should be a mechanism so that network \emph{a} and inform network \emph{b} of its nodes, and vice versa. Optimally this should already be a part of how nodes advertise their presence.
\item In order to take advantage of new nodes, they need to advertise their presence (and their location) to the entire system 
\end{itemize}

\section{Fundamental Routing Graph problems}


\section{Cluster methods}

\bibliographystyle{plain}
\bibliography{references}{}

\end{document}
