\documentclass[letter, 12pt, english, draft]{article}
\usepackage{mikkel}
\usepackage[ruled,vlined]{algorithm2e}

\title{Energy Affluent Mobile Ad-Hoc Networks}
\author{Mikkel Kjær Jensen}
\date{\today}
\begin{document}

\pagestyle{plain}

\section{Routing graph}
Having defined our problem in Section~\ref{fundamental}, I will in this section describe some of the different kinds of Routing graphs that can be used for \manet, and will describe the one I have used for this thesis. Furthermore I will give an introduction to single, second, and multi-level routing node organisation. 

In the literature I have read, I have not been able to find a consensus on how the optimal graph for \manet should be structured. It is furthermore my experience that the structure of the graph heavily influences the routing algorithm, making the two heavily dependant on each other.

\subsection{Single-level graphs}
The first important question to answer should be whether the graph should be planar or not. A planar graph is defined as a graph that can be embedded in the plane. One of the properties of a planar embedding is that no edge will cross another edge. As we saw in figure~\ref{fig:routing_not_planar} and \ref{fig:routing_planar}, that in the non-planar graph, each node can communicate directly with more nodes, compared to the planar graph. However, since it is harder to reason about non-planar graphs, I will chose only to work with planar graphs. \hide{\footnote{For instance if we let $v$ be the number of vertices's, $f$ the number of faces and $e$ the number of edges then $v - e + f = 2$. Thus we can begin to reason about the ratio of the different parts of the graph given at least one of them}, and many routing algorithms makes\todo{see if I can justify this}.}. This of course means that I must now take pains to ensure that the graph will remain planar for its lifetime.

I have found the following methods for making the planar graphs:

\begin{description}
\item[The \gabe:] Two nodes $n_1$ and $n_2$ only share an edge iff the circle that has $\overline{n_1n_2}$ as its diameter does not contain a node $n_3$ \cite{gopher}. 
\item[The \rng:] Two nodes $n_1$ and $n_2$ share an edge if there is no node $n_3$ such that $|\overline{n_1n_3}|, |\overline{n_2n_3}| \leq |\overline{n_1n_2}|$ (i.e. $n_3$ is closer to both $n_1$ and $n_2$ than $n_1$ and $n_2$ are to each other).
\end{description}

Both \gabe and \rng are based on the Unit graph assumption, and can therefore have problems in real world situations, accidentally removing edges that should otherwise have been there\cite{practical}.

\begin{description}
\item[The \cldp:] \cldp works by having each node uses \emph{the right-hand-rule}\footnote{The right-hand-rule states that for routing graphs, if you chose to go to the right each time you have a choice, you will eventually end up where you started.} and checks whether or not any of its links crosses/is crossed by another edge. \cldp does is not dependent on the Unit graph assumption \cite{practical}. 
\end{description}

\tikfig{graph_comparison}{\ref{fig:rng1} and \ref{fig:rng2} are examples of the RNG, while \ref{fig:gg1} and \ref{fig:gg2} are examples of the GG. This figure is inspired by figure~7 in \cite{GeoSpanners}.}

It is not immediately apparent how one would compare the groups created by these three algorithms. \cldp is especially hard to compare to the others given, that it is very much affected by the run-time behaviour of the transmission between the nodes. It is however clear that the \rng is slightly more accepting than the \gabe. See figure~\ref{fig:graph_comparison} for an example of the \rng and the \

Neither the \manet or the \anet can be said to be entirely static. Even in an \anet, where we assume the nodes remain stationary, every single node will eventually run out of energy or fail. The routing graph must therefore always have a maintenance system that can deal with changes to the underlying graph. 

To complicate matter further different nodes can have different communications radius as exemplified in figure~\ref{fig:node_different_radius}, which shows us that a node needs to confirm that it indeed has a bidirectional link to each of its adjacent nodes, and cannot take it for granted.

\tikfig{different_radius}{A node with different broadcast ranges. The first solid circle is the default broadcast range of node a, which does not reach node b. The second circle is the maximum range that a can broadcast, which does reach node b.}

Likewise it is important to remember that a node in practise might be able to adjust its communications range (at a higher energy cost). See figure~\ref{fig:different_radius}\todo{improve this}.

\subsection{Two-Level graphs}
\label{cluster methods}
Node Clustering is a hierarchical alternative to the flat graph. Instead of having every node being the same, a number of nodes are made into \ch such that all nodes in the graph has 1 cluster head that they communicate with. Most of the time the nodes that belong to a certain \ch is its one-hop neighbours, but there are also schemes that allows nodes to have a clustered that is farther away, in order to reduce the number of \ch. With \ch a layer specialisation and hierarchy is added to the network, which reduces the complexity of an otherwise entirely flat network, while still retaining some decentralisation.

In order to improve the routing, \ch are almost always accompanied by \emph{gateway nodes}. Gateway nodes are nodes that connect two clusters. \cite{spanners} defines a gateway node as follows:
Let C(p) be the set of nodes that has the cluster p as its \ch (this includes p). Then for a pair of \ch s $c_1, c_2$ if there exists a pair of nodes $p_1 \in C(c_1), p_2 \in C(c_2)$ such that $p_1$ and $p_2$ can communicate with each other in the graph, then $p_1$ and $p_2$ are gateway nodes (and in particular they are the only gateway nodes between $c_1$ and $c_2$.  

Some of the practical advantages are:
\begin{itemize}
\item Depending on the implemented location schemes, the \ch can cut down on the number of necessary transmissions needed to check whether nodes are alive or not.
\item It can make routing simpler, since all transmissions will have to go though the node gateways or up through the chain of \ch s. Only the \ch and gateway nodes needs to have their location database updated. All data will of course till have to be transmitted through regular nodes, but instead of having to route directly from the source to the sink, it will route from \ch to gateway, gateway to \ch and so on until it reaches the sink.
\end{itemize}

\ch is to divide any communication between nodes into 3 stages: 1) From the source to the local \ch, 2) From the local \ch to the \ch closest to the sink and 3) from the \ch closets to the sink to the sink itself\footnote{If either \ch is the sink or the source, and the first or third part is of course eliminated}. This is thus one way to decrease the number of nodes that needs to keep track of other nodes. Furthermore, if gateway nodes are employed, this can ease the routing problem, as a lengthy route is divided into shorter trips.  

However, since we are introducing a layer of centralisation, it is also clear that there must be a system in place to detect and replace a \ch should it fail (and to elect it in the first place). An often used system is a voting system, or an selection algorithm (based hash value of the node's id, its MAC value etc.).

In the case of \manet where there are some hardware in which some nodes are stationary, it might be preferable to use these stationary nodes as a subset of the \ch, since these nodes status as \ch would never change.

Since we are dealing with routing between \ch now, we can exploit this in the routing:
\begin{description}
\item[The \rdg:] The \rdg is a graph scheme that employs \ch (see Section~\ref{cluster methods}) based planar graph solution. After having clustered the nodes, each \ch calculates the local Delaunay edges, and then confers with its one-hop neighbours to check which Delaunay edges are valid. Besides being planar, computable in a distributed manner, the \rdg is also a spanner, which means that using it will only make the route a constant times longer than the optimal route \cite{GeoSpanners}.
\end{description}

\subsection{Multiple-level graphs}
A logical extension to the Two-level graph is the multiple-level graph. In the multiple-level graph the $[n]$th level contains the location of the nodes on the $[n-1]$th level, for $n \geq 3$. The selection for each level can be done similar to when dealing with second-level graphs, with redefined neighbours    
 \todo{find article and cite}

\section{Routing Algorithms}

The biggest conceptual difference for routing/shortest path algorithms in a \manet compared to main-stray routing/shortest path algorithms, is the loss of global overview that these algorithms employ --- such as relaxing a specific nodes' edges at a specific time. There exist distributed version of some of the ``classic'' routing algorithms, but their usefulness may be limited in a mobile network, since they will have to be recomputed either each time a node wants to transmit a message, or each time the underlying graph changes.

As mentioned in Section~\ref{fundamental}, the problem of sending a message from the source to the sink  becomes a graph routing problem (but not necessary a shortest-path routing problem, since factors like fairness or maximisation of the lifetime of the \manet becomes issues in practise).

When routing a message from the source to the sink, it is important to distinguish the message that is being transmitted and the nodes that are transmitting. The transmitted message does not only include the information that is intended for the node at the sink, but also extra meta-data needed for routing. This includes time-to-live, position of the source node, position of the sink, message id etc. The nodes are all assumed to use the same routing algorithm, and at least know who their neighbours are.

Most of the following algorithms will mostly be based around the idea of \emph{local routing algorithms}. 

Paraphrasing \cite{compass}, a local routing algorithm is an algorithm that for fills the following criteria:
\begin{enumerate}
\item Each message will have a limited memory to store information about a constant number of vertexes, but it have full access to the position of the sink and the source. At no point does the message have full knowledge of the entire graph. 
\item Let $s$ be the source of the message, $t$ the sink, and $v$ be a node in the graph that can be reached from $s$. Arriving at $v$, the message can then use the information $v$ has stored about its neighbours, as well as its logical edges to the neighbours. Based on this information, one of the neighbours of $v$ is chosen as the next receiver of the message, unless $v = t$, in which case the message will not be propagated.
\item A message will not leave information that the nodes routing algorithm will be able to use. In this sense nodes are considered stateless.\end{enumerate}

Taken together, the first and third property says that the state of routing process is contained in the message being routed, while the control and logic of the routing is contained in the nodes, which only acts upon the information in the message received and their knowledge about their neighbours. Since all the nodes are running the same routing algorithm, a well designed routing algorithm ensures that the message will arrive.

\label{record-recived}
It is worth noting that the 3 properties only apply to the routing algorithm itself, and not the layers below. This is an important distinction, as most routing solutions require reach node to track which messages it receives, so that future copies of the same message can be ignored in the future --- ensuring that a message will not needlessly congest the network. In the same vain, some routing solutions, such as \cite{speed} has its lower layers use this information to apply congestion avoidance. Another application is to use the messages being passed around to keep track of which nodes are still active --- this is usually combined with diagnostic or ``accounting'' messages\todo{Find ref}. 

\subsection{Categories of Routing algorithms}
When a routing algorithm wants to send a message or a number of messages from a sink to a source, an important question is whether there should first be sent a utility message to discover a path from the source to the sink, and then have all messages follow the established path, or each message should find its own way from the source to the sink.

If the algorithm chooses the first way, then there has been established two fundamentally different principles that can be applied.
\begin{description}
\route{Destination Sequential Distance Vector (DSDV)}
{With the DSDV, every node tries to remember as many as possible paths to the other nodes in the graph. The mapping from nodes to paths can either just be the next neighbour that a message needs to be routed to, or the entire path. In pure implementations of DSDV, each node knows a path to all other nodes. The advantage of DSDV is that when the information is correct it saves a lot of work, and is fast, while the disadvantage is that it is memory intensive and thus has problems scaling to large \manet. Depending on how much the nodes move, it could also be prone to contain outdated information.
%In order to find a path DSDV uses a distributed version of Bell
}
{DSDV}

\route{Ad-Hoc On-demand Distance Vector (AODV)}
{With the AODV, a node only records a path from any pair of nodes if it becomes relevant. This is done by a path discovery phase, where a route request package (RREQ) is sent from the source. When the RREQ arrives at a node it checks whether the node knows of a path to the sink. If the nodes does not, then the RREQ writes the node it was sent from and the number of hops to the current node, and is sent along. If the current node does know a path to the sink, then a bidirectional path is established from the source to the sink, so that messages can be exchanged.

The AODV has route maintenance mechanism. Let node $n_k$ and $n_{k+1}$ be the [$k$]th and [$t+1$]th node on the path from the source to the sink for $k \in \N$. If $n_{k+1}$ is no longer adjacent to $n_{k}$, then it will begin a tear-down protocol that will delete its routing table, and then tell the [$k-1$]th to do the same until the message arrives at the source, which can then rebuild the path if the path is still relevant.
AODV does not describe any specific routing algorithm in order to establish a route.}
{larMANET, AODV}

\end{description}

Routing algorithms suited for \manet s can be sorted into the following categories:
\begin{description}
\route{Flooding}
{The simplest of the routing algorithms. The entire network is flooded with the message, ensuring that the message eventually reaches the sink. We can be sure that the message will not flood the network forever, thanks to the recording system detailed in section \ref{record-recived}. If the sink can be reached from the source, then flooding is  guarantied to work, but uses a lot of bandwidth.}
{Sur2, larMANET} 

\tikfig{rdr}{The Grey area is the area in which routing take place. It is clear that the shape of the flooding will have a large effect on the number of nodes that will receive the transmission.}

\route{Restricted Direction Routing (RDR)}
      {Restricted Directed Routing is an algorithm that can be described as a kind of directed flooding. Let G be a geometrical shape. G is now placed such that it now contains both the source and the sink are both contained in G (preferably with a non-zero margin, to counter act any movement on the sinks part). The source now floods all nodes inside G, hopefully reaching the sink. Once a route to the sink has been established, the information

The there is no standard shape, but possible choices includes Linear Swept Sphere, Circle or Rectangle -- see figure \ref{fig:rdr}. Obviously, RDR is only applicable when the source has an idea of the location of the sink, otherwise another algorithm (like flooding) has to be employed. In the case where both nodes wants to contact each other (say, in situations where timers are involved), a bi-directional RDR can take place, where each node tries to find each other, until both diffusion reach each other, and a path can be established.}
{larMANET}

\route{Directed Diffusion (DD)}
{Directed Diffusion is a pull algorithm that works by having the sink send out a request for information. This requested information constitute a \emph{name} and can both be specific or be in a range\footnote{For instance it could be about data for a single day, or a range of days.}. This name, together with meta-data such as the sinks location and time-stamp, comprises an \emph{interest} which is transmitted to the rest of the network. Depending on the implementation, the request can flood the entire network, or just a section of it (like in RDR). When a node receives a request it checks with its data cache whether it has the information. If the node has the information, it sends a confirmation message back to the node it got the interest from. If the node does not have the information, it stores the interest in its own cache and creates a \emph{gradient} to both keep track of where the interest came from, have a timeout mechanism for that link, and a path maintenance mechanism (that also takes care of the interval between transmissions for that particular interest). 

Once a confirmation message arrives at the sink, the sink will perform \emph{reinforcement} that will use the established network of gradients to create one or more paths to the source, and begin to have the requested data transmitted over these paths. As a sink begins to receive data from a path, it will increase the flow of data from that particular path, leading to a positive feedback-loop that will eventually choose a single path over the others.

Since DD is a pull algorithm it might be especially useful in sensor networks.}
{directed}

\tikfig{greedy-problem}{In this topology a message is sent from \emph{a} to \emph{f}. The nodes that are connected with an edge have an bidirectional link, and the bidirectional links with a gray background are the ones that are routed through.We see that the algorithm first routes the message to node \emph{b}, since node \emph{b} is closer to node \emph{f} than node \emph{c}. However, since neither node \emph{a}, \emph{e} or \emph{d} are closer to node \emph{f} than node \emph{b}, the greedy routing stops, and the routing fails.}

\route{Greedy routing} 
{Greedy routing is a simple routing scheme that always chooses to route the message to that path of its neighbours that is closest to the sink. In the case where the message arrives at a node $n$, where all of $n$'s neighbours are farther away from the sink than $n$, then no action is specified, and the message will not be delivered -- see figure~\ref{fig:greedy-problem} for an illustration. The greedy algorithm thus offers no guarantee in delivering the message, but in practise it improves as the density of the network increases.}
{gopher, beyondUnit}

\route{Geometrical routing}
{Geometrical routing uses the location of the current node and the location of the sink. There exist many different kinds of routing algorithms, but common for them is that unlike the greedy routing technique, they will transmit the message to a node that is not necessarily the closer node. This ensures that the message will eventually reach the sink. An example of a Geometrical algorithm could be the OFR \cite{gopher} routing algorithm, which uses the line-segments between the sink and the source to find which faces in the graph it needs to route to eventually arrive at the sink}
{gopher+, gopher}

\route{Shortest path} 
      {The shortest path technique is to use distributed versions of the shortest path algorithms in order to find the shortest path from either one node to another, or from all nodes to all other nodes. As one might imagine, this approach has problems in mobile networks, where the structure of the graph can be in constant flux. Even in static networks the algorithms incur memory problems, since in the all-to-all solution, since for any node with a message for any node in the network must know which neighbour it must send it to get the most efficient route. Examples include Distributed Bellman-Ford}
{Sur2, contractionGraph, DSDV}

\end{description}

Different routing algorithms make different assumptions about the underlying nature of the mobility of the graph. There are several routing algorithms \cite{adaptive, two-tier} that assumes a number of permanently stationary nodes or sensors as a part of the infrastructure, while still having a number of mobile nodes that need to receive information. 

\tikfig{hidden_terminal}{The hidden terminal problem. In this example node c is communicating with node b (illustrated by the grey colour in node b's transmitting radius) Since the transmission does not reach node a, it is not possible for node a to detect the communication between node b and node c, and it will thus believe that it is free to transmit its message to node b. For more detailed information, see \cite{ComNet} or similar.}

In practical situations the system will have to deal with all the problems inherent in wireless communication, such as the hidden terminal problem (see figure~\ref{fig:hidden_terminal}) and message collision. To avoid these, having a lower communications layer that took care of this would be preferable to implementing it in the algorithm itself (though care should be taken to ensure that the algorithm is build to withstand these conditions).

\subsection{Fundamental Graph Routing problems}

A fundamental problem that has to be answered in any \manet is now to deal with situations where the sink no longer has the location the source routed the message to (but still reachable from the source). There are several solutions to this problem:

\begin{itemize}
\item Once the sink has moved a certain distance away from its neighbours, it could inform them of its current position, and then flood its surroundings of its current location. Effectively this solution creates an area of way-points that can be used to find the sink.
\item If we are dealing with a cluster method (see Section~\ref{cluster methods} for explanation of clusters), then the sink only needs to inform its old \ch that it has moved to a new \ch, and in the same way as before, the message will catch up with the sink.
\item Instead of getting the exact location of the sink at the beginning and then routing blindly towards that location, layers of node location ``servers'' can be established. A server may not know where any given node is, but it may know where a server (which is physically closer to the sink) that either knows the location of the sink, or knows a server closer to the sink \ldots and so on until we find a server that knows where the sink is. Thus the sink will eventually be found.
\end{itemize}

Another problem found in Graph routing is that of security. Unless some kind of security system is employed, a malicious node can not only impersonate other nodes, but it can also sniff the messages that are transmitted in its area (not only the ones that are directly sent to it). Further a malicious node can spam the network with bogus messages, both congestion the network, as well as draining the networks power. One proposed solution to this problem is to include secure keys that uniquely identify each node. However, without a central key-server (both for distribution and authentication), this is not practical.

\section{Problems we have with movement}

Movement presents us with the following problems:

\tikfig{shortest_path_change}{The nodes that are connected with an edge have an bidirectional link, and the bidirectional links with a gray background are the ones that are routed through. In \ref{fig:short_path1} we see a path from node $+$ to node $*$. In \ref{fig:short_path2} we clearly see that the movement of node $-$ has caused the routing protocol to find a much longer path to the node $*$ from node $+$.}

\begin{itemize}
\item Changes underlying graph -- see figure~\ref{fig:shortest_path_change}.
\item Invalidates location information about the node
\item If the node is part of path from \emph{a} to \emph{b} (and the node is neither), which is stored and which is believed to be used again, then that path may fail, and a longer path may have to be taken. In degenerate cases this may lead to situations where the new optimal path will lead away from the absolute coordinates of the end node.
\item The moving node may partition a network into 2 separate networks, if the node was the only node the 2 networks could communicate through
\item The moving node may create a link between 2 previously separate networks
\item The node may move into a position that creates a better path than the one before it moved - however, the surrounding nodes needs to be aware of it
\end{itemize}

The movement creates the need for the following solutions:
\begin{itemize}
\item There needs to be a way to restructure the underlying graph so that it reflects the movement of the node \cite{practical}
\item The location of the moving node needs to be updated. This does not necessary need to be done to all the nodes in the network, nor all the nodes that knew it before. There are many schemes for this, such as GLS \cite{scaleLocation} which partitions the world into squares, and then partitions these squares into other squares, resulting in several layers of squares. In each of these squares, there is a single node that knows the location of the node (and a scheme to find it).
\item Depending on implementation there may be no problem (there is a time-out system that detects that the message did not arrive, the node right before the missing node reports back that it could not deliver the message etc.). Finding a new path should not be a real problem unless the algorithm employed is a greedy algorithm (i.e. an algorithm that will always choose to go to the node that brings it closest to the absolute coordinates of the goal node).
\item If the network is partitioned, then the best course of action is to let the two networks figure out that the other part isn't available. Some kind of timeout mechanism in the location mechanism would seem like the most straightforward path
\item If connection is established between network \emph{a} and network \emph{b}, there should be a mechanism so that network \emph{a} and inform network \emph{b} of its nodes, and vice versa. Optimally this should already be a part of how nodes advertise their presence.
\item In order to take advantage of new nodes, they need to advertise their presence (and their location) to the entire system 
\end{itemize}

\section{Network simulators}

In order to test any algorithm I might implement, I will need a network simulator, to check how good it is compared to already existing algorithms. I have been able to identify 3 big network simulators: NS-2, GloMoSim and the Opnet Modeler.

\begin{description}
\item[NS-2:] NS-2 is a discrete event network simulator. The last version I have been able to find is from 2009. NS-2 uses the programming language Otcl for most of its non-core engine files -- such as routing algorithms and the scenario files that configures how the test is to be performed.
\item[GloMoSim:] GloMoSim is a parallel discrete-event simulator. GloMoSim is an old simulator. I have not been able   . GloMoSim uses the programming language \emph{Parsec}, a dialect of C, to write most of the non-core engine files -- such as routing algorithms. 
\item[Opnet Modeler:] Opnet is a commercial network simulator. Every layer of the network stack has to be modeled as a finite state machine \cite{MANcom}.
\end{description}

From \cite{MANcom} \todo{fix the references so it has the correct information} it is clear that the Network simulators gives widely different results. Since it is not clear which of the simulators which gives the result that is closest to reality, the parameters I will have to apply to find the best network simulator are: How many other articles use the network simulator\footnote{Since this means I would been able to compare it to more articles}, how flexible is it, how difficult is it to configure scenarios, how difficult is it to write new routing algorithms and how difficult is it to get a hold of.

Most of the articles I have read that have named their network simulator uses NS-2 [Find articles and cite them].

Both NS-2 and GloMoSim are very configurable, but NS-2 is more flexible than GloMoSim and has both more built-in features.

From what I have been able to derive from the manuals, and examples from both NS-2 and GloMoSim, GloMoSim is less complicated than NS-2.

Comparing NS-2 and GloMoSim it is clear that NS-2 is both the more complicated program to set up and program for. It is 

Both NS-2 and GloMoSim are free for education, while I will have to pay for Opnet.

Both NS-2 and GloMoSim has simple mobility models for the nodes built in. However, I have chosen a separate program, BonnMotion, which has been produced as a collaboration between the University of Bonn and the Toilers at \url{toilers.mines.edu} \todo{find a date where you last accessed it and write it -- perhaps this should be moved}. The program has multiple mobility methods built in -- such as ``Random Waypoint'', ``Manhatten Grid'' and 2 differen

\todo{Include tribute to bonnmotion and the toilers group}

\bibliographystyle{plain}
\bibliography{references}{}

\end{document}
