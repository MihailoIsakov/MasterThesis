\section{Routing graph}
\label{section:routing_graph}

Having defined the problem of routing in Section~\ref{section:fundamental_graph_routing}, I will in this section describe some of the different kinds of Routing graphs that can be used for \acp{manet}, and will describe the one I have used for this thesis. Furthermore I will give an introduction to single-, two-, and multi-level routing node organisation. 

In the literature I have read, I have not been able to find a consensus on how the optimal graph for \ac{manet} should be structured. From what I have read, it is clear that the structure of the graph heavily influences the routing algorithm, making the two heavily dependant on each other. 

\subsection{Single-level graphs}
\label{single_level}
The first important question to answer should be whether the graph should be planar or not. A planar graph is defined as a graph that can be embedded in the plane. One of the properties of a planar embedding is that no edge will cross another edge.
Curiously I have not been able to find any literature on the topic of Ad-hoc networks, which discuss the whether or not a graph should be planar, but rather they all a priori assume that the network must be planar, and then tried to find the solution based on that. I will however in the following try to debate both cases, hopefully giving the reader better insight why, or why not to use planar graphs.\todo{work on placement}

 As we saw in figure~\ref{fig:routing_not_planar} and \ref{fig:routing_planar}, that in the non-planar graph, each node can communicate directly with more nodes, compared to the planar graph. For an illustration of this see figure\todo{make planar example figure} . The question then becomes whether the graph we want to use should be planar or not. The advantage of having a non-planar graph is that there might be a faster route from the source to the sink (since more edges are allowed). A non-planar graph, is however problematic from a theoretical and practical perspective.

From a theoretical perspective it is clear that we know less about a non-planar graph than that of a planar one. For instance we have Euler's formula, which states that if we let $v$ be the number of vertices's, $f$ the number of faces and $e$ the number of edges then $v - e + f = 2$. Thus we can begin to reason about the ratio of the graph if we are given one of $e$, $f$ or $v$. This also becomes a problem with routing, as it likewise becomes harder to reason about routing with non-planar graphs, and most traditional routing algorithms assume to work on planar graphs. It is also important to remember that a non-planar graph will contain more edges that has to be taken into account.

From a practical perspective working with non-planar graphs will mean that each node will have to keep track of more neighbours (since there is no theoretical upper bound, and in practise greatly increases based on the radius of the node\footnote{Realistically I would however say that more than 40 neighbours would be unrealistic, even for crowded areas.}). In dense networks the extra neighbours added by having a non-planar graph can be significant --- see figure \todo{make crowded street example} for an illustration of this. The number of neighbours for each node also becomes a problem with routing schemes that uses geometrical information, as more nodes locations have to be inspected, than if we had a planar graph. It should be noted, however, that while more energy has to be used to keep track of neighbours, this does not necessarily mean that there will be a greater transmission overhead, as we are assuming the nodes are transmitting their location and status through radio waves (which propagate within a certain radius), which would have to be sent in any case.

From this I conclude that while a non-planar graph can be used as an alternative to planar graphs, I do feel that the loss of knowledge and extra bookkeeping is worth it. I will therefore in this thesis only pursue planar graphs \todo{Try to include analysis of the graphs here}.
 
Having chosen to keep our graphs planar we must take care that the routing graphs will always be planar.

I have found the following general methods for making the planar graphs:
\begin{description}
\item[The \ac{gabe}:] Two nodes $v$ and $u$ only share an edge iff the circle that has $\overline{vu}$ as its diameter does not contain a node $q$ \cite{gopher, gpsr}. 
\item[The \ac{rng}:] Two nodes $v$ and $u$ share an edge if there is no node $q$ such that $|\overline{vq}|, |\overline{uq}| \leq |\overline{vu}|$ (i.e. $q$ is closer to both $v$ and $u$ than $v$ and $u$ are to each other). \cite{gpsr, RNG}
\end{description}

Both \ac{gabe} and \ac{rng} are based on the \ac{uga}, and can therefore have problems in real world situations, accidentally removing edges that should otherwise have been there\cite{practical}.
\label{right-hand rule}
\begin{description}
\item[The \ac{cldp}:] \ac{cldp} works by having each node uses \emph{the right-hand-rule}\footnote{The right-hand-rule states that for routing graphs, if you chose to go to the right each time you have a choice, you will eventually end up where you started.} and checks whether or not any of its links crosses/is crossed by another edge. In other words, \ac{cldp} first builds up a graph that may be non-planar, and then in a distributed manner finds and removes edges that makes the graph non-planar without disconnecting parts of graph (as the paper accuses both \ac{gabe} and \ac{rng} of).\ac{cldp} does is not dependent on the Unit graph assumption \cite{practical}. 
\end{description}

The \ac{gabe} is a sub-graph of the Delanay triangulation, and \ac{rng} is a subset of \ac{gabe} \todo{find and cite}, and as such they are both well described and understood graphs many of the same properties as the Delaunay triangulation. \ac{cldp} on the other hand is not a graph, but rather a dynamic algorithm for creating planar graphs that according to \cite{practical} are less disconnected than graphs made using \ac{gabe} and \ac{rng}. It is therefore harder to reason about most probable maximum number of neighbours for \ac{cldp}, as it is less of a graph and more of a process that creates a graph. 

\tikfig{graph_comparison}{\ref{fig:rng1} and \ref{fig:rng2} are examples of the \ac{rng}, while \ref{fig:gabe1} and \ref{fig:gabe2} are examples of the \ac{gabe}. This figure is inspired by figure~7 in \cite{GeoSpanners}.}

Delaunay graphs have been found to have an expected maximum degree of $\theta(\log n / \log \log n)$ \cite{delExpected}, and furthermore that that \ac{gabe} has a maximum degree of $\theta(\log n / \log \log n)$, or, as the authors put it, there exists a constant $a$ and constant $b$ such that 
$$
\lim_{n \rightarrow \infty} Pr{\Delta \in [\frac{a\log n}{\log \log n}, \frac{b\log n}{\log \log n}]} = 1
$$
and for \ac{gabe} $a = 1/12$ and $b = 1$, for uniform distributions \cite{GGExpected}.

It is not immediately apparent how one would compare \ac{gabe} and \ac{rng} to \ac{cldp}, since \ac{cldp} is effected on the run-time behaviour of the transmission between the nodes. However, as mentioned before the \ac{rng} is a subset of the \ac{gabe}, which illustrated in figure~\ref{fig:graph_comparison} and figure~\ref{fig:gg-rng-example-big}.
\hide{
\tikfig{gg-rng-example-big}{In the following examples all the nodes have the same transmission radius. In figure~\ref{fig:norm_graph} we see the non-planar graph. In figure~\ref{fig:gg_graph} we see the \ac{gabe} for the same set of points, clearly being planar and with far less edges, and in figure~\ref{fig:rng_graph} we see the \ac{rng} graph, which on closer inspection turns out to be a subset of the \ref{fig:gg_graph}. For comparison we have the Minimum Spanning tree in figure~\ref{fig:mst}}\todo{unhide when we have to print}
}
Neither the \ac{manet} or the \ac{anet} can be said to be entirely static. Even in an \ac{anet}, where we assume the nodes remain stationary, every single node will eventually run out of energy or fail. The routing graph must therefore always have a maintenance system that can deal with changes to the underlying graph. 

To complicate matter further different nodes can have different communications radius as exemplified in figure~\ref{fig:node_different_radius}, which shows us that a node needs to confirm that it indeed has a bidirectional link to each of its adjacent nodes, and cannot take it for granted.

\tikfig{different_radius}{A node with different broadcast ranges. The first solid circle is the default broadcast range of node a, which does not reach node b. The second circle is the maximum range that a can broadcast, which does reach node b.}

Likewise it is important to remember that a node in practise might be able to adjust its communications range (at a higher energy cost). See figure~\ref{fig:different_radius}\todo{improve this}.

\subsection{Two-Level graphs}
\tikfig{gateway-node}{Node $s$ and $t$ are the source and the sink respectively, the rectangular nodes with $CH$ are the \acp{ch} and the diamond nodes are the gateway nodes. From the image we can see that the message travels from the source to the header, between the gateways and then to the second  and then lastly to the sink. It is also clear that it would have been faster to go from the gateway and then to the sink, but since the gateway in this model does not know the location of the different nodes, this is impossible in this model.}

\label{cluster methods}
Node Clustering is a hierarchical alternative to the flat graph. Instead of having every node being the same, a number of nodes are made into \acp{ch} such that all nodes in the graph has 1 cluster head that they communicate with. Most of the time the nodes that belong to a certain \ac{ch} is its one-hop neighbours, but there are also schemes that allows nodes to have a clustered that is farther away, in order to reduce the number of \acp{ch}. With \acp{ch} a layer specialisation and hierarchy is added to the network, which reduces the complexity of an otherwise entirely flat network, while still retaining some decentralisation.

In order to improve the routing, \acp{ch} are almost always accompanied by \emph{gateway nodes}. Gateway nodes are nodes that connect two clusters --- see figure~\ref{fig:gateway-node} for an illustration. \cite{spanners} defines a gateway node as follows:
Let C(p) be the set of nodes that has the cluster p as its \ac{ch} (this includes p). Then for a pair of \ac{ch} s $c_1, c_2$ if there exists a pair of nodes $p_1 \in C(c_1), p_2 \in C(c_2)$ such that $p_1$ and $p_2$ can communicate with each other in the graph, then $p_1$ and $p_2$ are gateway nodes (and in particular they are the only gateway nodes between $c_1$ and $c_2$.  

Some of the practical advantages are:
\begin{itemize}
\item The \ac{ch} will cut down on the number of necessary transmissions needed to check whether nodes are alive or not.
\item \acp{ch} makes routing simpler, since all transmissions will have to go though the node gateways or up through the chain of \acp{ch}. Thus only the \acp{ch} and gateway nodes will need to update their location database. Messages will still have to be sent through regular nodes (as illustrated in figure~\ref{fig:gateway-node}), but instead of having to route directly from the source to the sink, the message will alternating be routed through \acp{ch} and gateways, until it reaches the sink.
\end{itemize}

Using \acp{ch} we divide the communication between nodes into 3 stages:
\begin{enumerate}
\item From the source to the local \ac{ch} 
\item From the local \ac{ch} to the \ac{ch} closest to the sink
\item From the \ac{ch} closets to the sink to the sink itself\footnote{If either \ac{ch} is the sink or the source, and the first or third part is of course eliminated}.
\end{enumerate}\todo{refernce the figure of this}
 \acp{ch} thus decreases the number of nodes that needs to keep track of the other nodes, and the gateway nodes divides the routing into multiple smaller trips.  

However, since we are introducing a layer of centralisation, we must ensure that we can detect and replace a \ac{ch} should it fail (and to elect it in the first place). An often used system is a voting system, or an selection algorithm (based hash value of the node's id, its MAC value etc.).

In the case of \acp{manet} in which some nodes are stationary (such as \cite{two-tier}), it might be preferable to give these stationary nodes a permanent status as \ac{ch}, as this will at best cut down on the number of elections needed to establish \acp{ch} (since the nodes near them will accept them as their \ac{ch}), and at worst not change anything (in the case none of mobile nodes have them within their transmission range).

One way of exploiting routing between \acp{ch} is to use the \ac{rdg}:
\begin{description}
\item[The \ac{rdg}:] The \ac{rdg} is a graph scheme that employs \acp{ch} based planar graph solution. After having found the set $C$ of \acp{ch}, each $ch \in C$ calculates its local Delaunay edges on the set $C$, and then confers with its one-hop \ac{ch} neighbours to check which Delaunay edges are valid. Besides being planar and computable in a distributed manner, the \ac{rdg} is also a spanner, which means that using it will only make the route a constant times longer than the optimal route \cite{GeoSpanners}.
\end{description}

\subsection{Multiple-level graphs}
A logical extension to the Two-level graph is the multiple-level graph. In the multiple-level graph the $[n]$th level contains the location of the nodes on the $[n-1]$th level, for $n \geq 3$. The selection for each level can be done similar to when dealing with second-level graphs, with redefined neighbours    
 \todo{find article and cite}

\subsection{Spanners}
\label{section:spanners}
As mentioned in \ref{section:graph_concepts} it would be very nice if we could get a guarantee on how effective a given graph building algorithm would be. However, before looking further into this we must first decide what a spanner means for a \ac{manet}.

When dealing with pure graph theory, as long as the space is well-defined, it is always clear what the shortest path between two nodes in the graph. For instance, with a non-planar graph on the planer under the euclidean metric, the shortest path will always be a straight line, so the graph with the lest distance will be the complete graph. Using the complete graph is of course inefficient, and multiple algorithms and graph types have been invented to find an acceptable trade-off between the number of edges, extra auxiliary nodes, and the distance between any two nodes.

When trying to apply spanners to \acp{manet} we have to remember that each node has a limited broadcast range, creating an upper bound on the length of an edge. Depending on the scheme and the hardware of the node, each node may cost same, no matter the length of the edge (since each message is not directed, but rather transmitted omni-directionally). I will therefore in the following use the term limited-range spanner when dealing with spanners for \acp{manet}. 

\todo{make figure of limited-range spanner}

Due to lack of empirical data I have tested the   500 point-sets with 1000 nodes in a 500 by 500 plane and a transmission range of 20. I have tested the spanner by producing 3 different graphs: The non-planar maximum graph \todo{define this before this point}, the \ac{gabe} and the \ac{rng}. For each point-set I have generated 10 different pairs of nodes, and then checked the distance and number of nodes between them (both having the weight of the edges being proportional to their length, but also where each edge costs the same --- the unit graph) for all three graphs. 

I will use the non-planar maximum graphs as the benchmark that I will hold the other two graph types up against. I will record shortest, longest and average number of nodes the message has to be transported through, and the number of lost paths (i.e. paths that existed in the more complete graph of the point-set, but isn't present in this graph). 

Since there is no guarantee that two nodes will be connected in any of the graphs, I will have to deal with such cases.

If there is no path between a pair in the non-planar graph, then there cannot be a path between the pair in either the \ac{gabe} or the \ac{rng}, as these are sub-graphs of the non-planar graph, I will ignore it. However, if there is a path in the path between a pair in the non-planar graph, but none in either the \ac{gabe} or the \ac{rng}, then I will record a miss for that graph, since it is an obvious detriment. 

The python scrips I used to generate and tests the graphs can be found in the \texttt{src/spanner/spanner\_check.py} along with the data (see the README for explanation) and \texttt{src/spanner/graph\_support} in the GitHub repository refereed to in section~\ref{section:source_code} on p.~\pageref{section:source_code}. All requirements in the code is mentioned in the README file.\\

The non-planar maximum graph:
\begin{tabular}{ccrrrr}
\multicolumn{2}{}{}        & Length of graph: & Max node-pair: & Min node-pair: & Avg node-pair\\
\multirow{3}{*}{Distance}  & NML & 2843864.90 & 200.17 & 1.00 & 56.88\\
                             & GG  &  3111039.29 & 221.81 & 1.00 & 62.22\\
                            & RNG & 3555742.22 & 245.12 & 1.00 & 71.11\\
 \hline 
Unit      & NML & 190687\phantom{.00} & 13.00 & 1.00 & 3.81\\
Distance  & GG  & 339200\phantom{.00} & 27.00 & 1.00 & 6.78\\
          & RNG & 443490\phantom{.00} & 34.00 & 1.00 & 8.87\\
\hline
\hline
               &     & Distance:   & Unit Distance: & \multicolumn{1}{||c}{} &  \# Missing path \\
Percentage     & NML & 100.00 \% & 100.00 \%    & \multicolumn{1}{||c}{NML} &  0 \\
compared to the& GG  & 109.39     \% & 177.88 \%        & \multicolumn{1}{||c}{GG} &  0 \\
normal graph   & RNG & 125.03     \% & 232.57 \%        & \multicolumn{1}{||c}{RNG} &  0 \\
\hline
\# Connected Components: & 1.084 
\end{tabular}\\

The Gabriel Graph:
\begin{tabular}{lll}
 & Distance & Unit Distance\\
Total: & 269058.0535 & 40647\\
Average: & 10.2389 & 1.5468\\
Min value: & 1.0000 & 1\\
Max value: & 59.7385 & 8\\
\hline
Number of missing paths: & 0 &\\
\end{tabular}\\
Number of edges: 15\\
Average numbers of neighbours: 0.31\\
Minimum number of neighbours: 2\\
Maximum number of neighbours: 0\\
\\

The RNG:
\begin{tabular}{lll}
 & Distance & Unit Distance\\
Total: & 275979.7471 & 42179\\
Average: & 10.5023 & 1.6051\\
Min value: & 1.0000 & 1\\
Max value: & 61.1101 & 10\\
\hline
Number of missing paths: & 0 &\\
\end{tabular}\\
Number of edges: 15\\
Average numbers of neighbours: 0.31\\
Minimum number of neighbours: 2\\
Maximum number of neighbours: 0\\


It is very curios to note that all the reults are exactly the same. Unless we are dealing with an error in the scripts making it, we must conclude that we in we practice does not have a longer route if we use a planar graph instead of a non-planar maximum graph.
