% [Conclusions are very important. Do not expect that the reader remembers everything you told him/her.
% Having stated the definitions, you can now be more specific that in the introduction]
% * Overview what this work was about.
% * Main results and contributions
% * Comments on importance or
% * Tips for practical use [how your results or experience can help someone in practice or
% another researcher to use your simulator or avoid pitfalls]
% * Future work. Reinforce the importance of work, but avoid giving out your ideas].

\section{Conclusion}
\label{section:conclusion}

\subsection{Summery}

I have in this project:
\begin{itemize}
\item Described the \ac{manet} and given several real-world applications.
\item Described the elements needed to build a working \ac{manet}, as well as describing many of the decisions that needs to be taken when designing a new \ac{manet} solution.
\item Described many of the problems that are introduced by the ad-hoc, decentralised nature of the \ac{manet}, and given my own ideas on how to fix these issues.
\item Described several of the different categories of routing algorithms, as well as having given the algorithm for GOAFR, GOAFR+, Greedy routing and GPSR.
\item Described the \ac{gabe} and the \ac{rng} and made an extensive comparison to the non-planar graph and each other.
\end{itemize}

\subsection{Results}
\begin{itemize}
\item I have made an implementation of GOAFR for the ns-2 network simulator and tested it favourably compared to GPSR and the Greedy routing algorithm.
\item I have shown and argued that in practice cannot conclude whether or not uniformly distributed nodes are limited-range spanners, they are both so very close on average, up to at least 10000 nodes, that there in practice may be no difference, and that \ac{gabe} and the \ac{rng} therefore are good choices for graphs with uniformly distributed nodes.
\item I have made an comparison between the GOAFR, GPSR, Greedy and the DSDV, and shown that GOAFR  and GPSR clearly are much more effective than both the Greedy and the DSDV routing algorithm.
\item I have created an all-around introductory guide to the required components for a \ac{manet}, and included references to several articles and papers that should enable the reader to begin his own research, if so inclined.
\item Shown that for uniformly distributed nodes the \ac{gabe} and the \ac{rng} will on average have about 3.5 to 3.8 and around 2.5 neighbours respectively, far better than the non-planar graph, which averages around 12.
\item I have successfully learned how to use the ns-2 network simulator to create traces that can be used to asses the quality of a routing algorithm.
\end{itemize}

\subsection{Future work}
\label{section:future_work}
It would be interesting to see an analysis of the Non-planar, the \ac{gabe}, the \ac{rng} as well as the \ac{rdg} on graphs created by node clustering algorithms or mobility models such as the Nomadic Community Mobility Model, Pursue Mobility Model and Reference Point Group Mobility Model, Gauss-Markov \cite{MobilityAdHocResearch} and the Disaster Area mobility model \cite{disasterArea}.

I would suggets that a survey should be made on data traffic models, and the software that could generate them, as well as having them better integrated with the papers dealing with network simulators. As mentioned in section~\ref{section:traffic_model}, the ns-2 manual does not have much to say on the subject, and neither does the manual for GloMoSim. This would be an easy, and logical, way to increase the presence and use of traffic simulators.

Given the existence of affordable mobile nodes with transmitters in the form of smart-phones (and an open software stack in the case of phones running the Android OS), it would seem obvious to conduct empirical field tests with implementations of the various routing algorithms and graph implementations, in order to test the theory in real world situations and settings.  

