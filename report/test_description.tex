\section{Test description}
\label{section:test_description}

In this I will describe the tests, what kind of metrics I will record, how the tests are structured and their parameters.

\subsection{Routing algorithm tests}
\label{section:routing_algo}

\subsubsection{Metrics}

The metrics I will measure in the tests will be the following
\begin{description}
\item[Number of hops:] The number of hops from the source to the sink. For this I will measure the maximum number of hops, the minimum number of hops, as well as the average. Since each hop made means that a new node will have to expend energy and processing power, which will also greatly influence the time it will take for the message to arrive at its destination, and increase the risk is that the network topology will make it impossible for the message to arrive.

\item[Time:] The amount of time spent sending the message from the source to the sink. 

\item[Percentages of successfully arrived messages:] While most routing algorithms guarantee that the message will always arrive, this is not always the case. The best example of a routing algorithm that does not have this guarantee is the greedy routing algorithm (see section~\ref{section:greedy}), and I therefore feel this is a very interesting metric.
\end{description}

For the number of hops and the amount of time the maximum, the minimum and the average value is recorded, and the standard deviation will be calculated. The main value for comparison will be the average value, but the others will be interesting for to identify outliers and see how representative the values are.

\subsubsection{Input parameters}
\label{section:input_parameters}
The parameters and their values for the different tests are as follows:
\begin{description}
\item[Routing Algorithms:] Since the purpose of this test is to test different routing algorithms, we must pick a subset of the available algorithms. I have chosen the following algorithms: GOAFR \cite{gopher}, GPSR \cite{gpsr}, Greedy routing \cite{gopher}, and DSDV \cite{DSDV} --- a mix of several categories (see section~\ref{section:cat_routing} and section~\ref{section:routing_algorithm} for more about these algorithms).

\item[Planarity algorithm:] GPSR, GOAFR and Greedy will all be running on a \ac{gabe} graph, while the DSDV will be using the \ac{ucg} (see~\ref{}).

\item[Movement model:] The topology is clearly going to be influenced by the way that the nodes move, therefore it would be interesting to test out several models, to see the strengths and weaknesses of the different routing algorithms. In practice I will use  the Gauss-Markov \cite{MobilityAdHocResearch} with the trace created through the use of the BonnMotion tool \cite{toilers} use to create the movement files which can then be imported into ns-2\footnote{For the script that generates the movement files, please see \texttt{src/Motion/MotionCode/Gauss-Markov.py.}}. The nodes will move for 120 seconds before the recording of their locations will begin, introducing a greater element of randomization. See section~\ref{section:mobility_model} for more details about Gauss-Markov.

\item[Amount and data-transmission type:] In real world situations there will be different levels of traffic on the network, and therefore to test it we must likewise simulate these differences. The choice of transmission protocol is also important. In practice I will try to transmit 512 kB through a TCP connection. In order not to swamp the network in traffic, only 33\% of the nodes will ever communicate. See section~\ref{section:traffic_model} for more information on how they were generated.

\item[Size of the simulation area:] The size of the simulation area has a direct influence on how far apart the nodes can move and how many nodes are needed for a given node density. Everything else being equal, a larger simulation area will make the network less robust. In practice I will create three differently sized simulation areas, one that is 100 by 100 units, 500 by 500 units and one that is 750 by 750 units.

\item[Number of nodes:] It is clear that denser node distribution, everything else being equal, will give a more robust network. In practice I will have node-sets of 10 to 100 nodes in increments of 10 nodes, giving me 10 data-points that will be used to analyse the success of the different routing algorithms.

\item[Time simulated:] There is a clear need to balance the time we want to simulate with the time it will take to execute. I have found 150 seconds to be a good compromise (since the last transmissions also have to come through) --- this figure will be used both to generate the traffic as well as the mobility.

\item[Radio distance:] The distance the radio can transmit is going to have a large impact on how well connected the graph is going to be, and, everything else being equal, how many hops that are required to go from the source and the sink. I have however not been able to change the radio range for this simulation, as doing so would have required that I calculated the reciving threshold using the included file \texttt{threshold.cc}\footnote{\texttt{threshold.cc} can be found in \texttt{ns-2.33/indep-utils/propagation} directory of the ns-allinone-2.33 folder.}. While the reported error seemed fixable, I was afraid that it was a symptom of bigger problems choose not to attempt to fix it. Instead I went with the default 250 units radio range. 
\end{description}


\subsubsection{Routing algorithm description}
\label{section:routing_algorithm}
In this description of the chosen routing algorithms, I will assume by default that there exists a path from the source to the sink, and if needed I will detail how the algorithm handles the cases where no such path exists.

Since I have already explained how the Greedy routing works in \ref{section:greedy}, I will not repeat it here.

\paragraph{GOAFR and GOAFR+}

\tikfig{gopher}{Here we see a representation of the GOAFR with its ellipse. In this example $s$ and $t$ are the nodes that wants to communicate, and therefore acts as the forci for the ellipse. It is clear from this image that the ellipse is quite large (and in fact almost circular) compared to the distance between the source and the sink. From this it should be easy to see that for dense graphs there is not much of a chance for the message to go outside the ellipse due to the large amount of wiggle room, while still being able to avoid bad edge conditions.}

GOAFR (Greedy Other Adaptive Face Routing) \cite{gopher} and GOAFR+ (a refinement on GOAFR) \cite{gopher+} are multi-stage geometrical routing algorithms. The algorithms are based on Greedy routing (see section~\ref{section:greedy}) and Adaptive Face Routing, with an adaptive ellipse $\varepsilon$ as the boundary for the message. The description in the following is based on \cite{gopher}.

As an integral part of the algorithm, GOAFR includes an ellipse $\varepsilon$ that the message has to inside, in order to avoid having the message going astray. $\varepsilon$ is created so that its forci are the source $s$ and the sink $t$, and so that its semi-major axis starts out being $2 * |\overline{st}|$. While it is not detailed in \cite{gopher} it is likely that the ellipse was chosen since it is inheiriently based on two points in the plane (the forci), the major and minor axis can be expanded independtly of each other. 

Finding out whether a point is inside an ellipse is also very easy. If we let $x$ be the node we want to check, and $p, q$ be the foci of the ellipse, and $j$ is the semi-major axis (major radius) then $x$ is inside contained in the ellipse iff 
\eqn{
\overline{|xp|} + \overline{|xq|} \leq j
}

Using the formula for forci
\eqn{
F = \sqrt{j^2 - n^2}
}
where $F$ is the distance from each forci to the centre, $j$ is the semi-major axis as before, and $n$ is semi-minor axis (minor radius) we have
\eqn{
\frac{|\overline{st}|}{2} = \sqrt{j^2 - n^2} \Leftrightarrow n = \sqrt{j^2 - \left(\frac{|\overline{st}|}{2}\right)^2}
}
which we then can use to calculate the semi-minor axis. For an illustration of the ellipse, see figure~\ref{fig:gopher}. Depending on the execution of GOAFR the major axis of $\varepsilon$ might grow, as detailed in the next two paragraphs. 

Having thus laid the groundwork I can now describe GOAFR. Unless otherwise specified, once a step has been comleted, then algorithm will go to the next step.
\begin{enumerate}
\item Make the ellipse based on the sink and the source.

\item Perform Greedy routing as described in section~\ref{section:greedy} on p. \pageref{section:greedy}. If the greedy algorithm sends the message beyond the bound of $\varepsilon$, then the longest axis of $\varepsilon$ is doubled. If the message arrives at the sink, then the message is handed over to the node and the algorithm terminates. Otherwise we have reached a local minimum (see section~\ref{section:greedy} for an explanation). 

\item Since we have arrived at a local minimum $u$, we first store $u$ as both $u_{local\_min}$ and as $u_{closest\_node}$ in the message header, to indicate that it is both the node we started routing around the face as, but also the node that is closest to the sink. We then use Adaptive Face routing (which uses the right-hand rule --- see section~\ref{right-hand-rule}) to traverse the face, and for each node $v$ we encounter we check to see if it is closer to the sink than the current value of $u_{closest\_node}$, if so, we set $u_{closest\_node}$ to be $v$. 

If the message is about to go outside of $\varepsilon$ for the first time, then the message reverses direction and the message tries to traverse the face in the other direction (skipping past $u_{local\_min}$) without stopping). If the message encounters $\varepsilon$ for the second time, then the longest axis of $\varepsilon$ is doubled and the message is forwarded as normal.

At some point the message will either arrive back at $u_{local\_min}$ or find the sink. If the message arrives at the sink, then the message is handed over to the sink and the algorithm terminates. If the message arrives back at $u_{local\_min}$ after having gone outside $\varepsilon$ an even number of times, we go to the next step.

\item The message is now forwarded to $u_{closest\_node}$ we found in the last step, by using the same face routing as before. Once we arrive at $u_{closest\_node}$ we go to step 2.
\end{enumerate}

GOAFR+ works similarly to GOAFR, with the difference being the introduction of two variables $m$ and $n$. While doing the adaptive face routing from the local minimum node $u_{local\_ min}$, the algorithm keeps track of how many nodes encountered that are closer to the sink than $u_{local min}$ in the variable $m$ and how many nodes are farther away from the sink than $u_{local\_min}$ in the variable $n$.
A special action is then taken if
\begin{enumerate}
\item If the message arrives at the edge of $\varepsilon$ for the first time, then the message turns back and explores the face in the other direction. 
\item If the message arrives at the edge of $\varepsilon$ for the second time, and no nodes have been found to be closer to the sink than $x$, then the major axis of $\varepsilon$ is doubled. Otherwise the message is sent to the node closest to the sink and the greedy routing is resumed.
\item If at any point $m > \rho n$, where $\rho$ is a constant defined before routing starts, then route the message to the node closest to the sink and begin greedy routing again.
\end{enumerate}

Both GOAFR and GOAFR+ have been proved that if $|p^*|$ is the distance of the optimal Euclidian path, and $c$ is the fixed major axis length of the elipse, then GOAFR will reach the sink after traversing at most $O(c^2)$ edges and will at most have a cost of $O(c^2(p^*))$ for distance. From \cite{asymp} we learn that any geometric routing algorithm has an edge cost of  $\Omega(c^2)$, and therefore GOAFR is both theoretical asymptotically worst-case optimal and average-case efficient. Due to space demands I will not try to prove these claims here, but will instead refer the interested reader to \cite{gopher+, asymp, gopher}.

Due to time constraints I have only been able to implement GOAFR in ns-2, and so will only use that in the tests.

\paragraph{GPSR}
\tikfig{gpsr}
{A demonstration of the GPSR routing algorithm. The grey background with grey nodes indicates when the GPSR algorithms is in the Greedy phase and the black nodes with the black background indicates when the algorithm is running in perimeter phase. In this case there is no switch of parameters.}

\label{section:gpsr}
Greedy Perimeter Stateless Routing is a routing algorithm that uses a hybrid of Greedy routing and face routing. The idea is to use Greedy routing as much as possible, and only use the face routing recover from situations where it reaches a situation where the Greedy routing fails. The algorithm will use the right-hand-rule (see section~\ref{right-hand-rule}) to route the messages in perimeter mode to ensure they stay within the same face. The description here is based on \cite{gpsr}.

If we let $x$ be the point where the Greedy routing fails, and $t$ be the sink, then the gist of the perimeter routing is to route through the faces that are intersected by the line-segment $\overline{xt}$, in order to find a point that is closer to $t$ than $x$, and possibly recover. In order to do this the algorithm will explore each face sequentially, moving on to the next face when it encounters a node or, more likely, an edge that is intersected by $\overline{xt}$. For an example of routing see figure~\ref{fig:gpsr}. 

In order for the routing to work the algorithm relies on 4 values:
\begin{description}
\item[$t$:] The sink, which contains its own coordinates
\item[$L_p$:] The node where the algorithm went from Greedy routing to Perimeter routing. This value is overwritten if the routing goes back to Greedy routing and then back to Perimeter routing. This, together with $t$, is used to create the line-segment $\overline{L_pt}$, that is used to signal when the message is close enough to $t$ to move to the next face.
\item[$L_f$:] The point on $\overline{L_pt}$ where the message entered the current face. Most of the time this will not be the position of a node, but rather a point on an edge between the face whose perimeter the message has just explored, and the next face. 
\item[$e_0$:] The first edge the message traversed on the current face. This is used as the termination condition for the perimeter exploration in the case where $t$ is unreachable. 
\end{description}

Also, in the following let $f$ denote the current face we are routing on. Unless otherwise specified, the algorithm will always go to the next step once it has completed a step.

\begin{enumerate}
\item The algorithm begins with greedy routing until it either reaches the sink, in which case it terminates, or it reaches a local minimum and therefore fails. In the case of failure the message enters perimeter mode.

\item The message will begin perimeter routing and record its current node as both $L_p$ and $L_f$, and store the line-segment $\overline{L_pt}$, which we will use throughout our routing. The edge counter-clockwise to $L_p$ is stored in $e_0$.

\item Let $p$ be the current node. If $|\overline{pt}| < |\overline{L_pt}|$ we have reached a node closer to the sink than the local minimum, and we continue with our Greedy routing --- the algorithm goes to step 1. If on the other hand $p$ is farther away from the sink than $L_p$, the message is transferred to the counter-clockwise node $q$ on $f$. If $\overline{pq}$ is the same edge as $e_0$ then we have come full circle, and there exist no path to the sink $t$, so drop the message. If on the other hand either $q$ or $\overline{pq}$ is intersected by $\overline{L_pt}$, then transmit the message to $q$ and go to step 4, otherwise just transmit the message and go to step 3.

\item Let $p$ be the current node. Set $L_f$ to $p$ and switch to the next face. Let $q$ be the counter-clockwise neighbour of $p$ on the new face and set $e_0$ to $\overline{pq}$. Go to step 3. 
\end{enumerate}

I have not been able to find any real theretical results in either \cite{gpsr} or any of the literature I have read.

\paragraph{DSDV}

As specified before DSDV uses a distributed version of Bellman-Ford routing algorithm to find the fewest number of hops between itself and any other node. For this end, each node retain a routing table, with now and again is refreshed. The only theoretical result I have been able to find is that DSDV garuntees loop-free routing \cite{DSDV}.

\subsubsection{Mobility models}
\label{section:mobility_model}

\tikfig{gauss-markov}{An example of 10 nodes moving around. Each node has its own colour trail to differentiate it from the others, and to show the path of the node --- the grey nodes on the trail are the various locations they were ordered to go to.}

For the mobility models I have only chosen to test the entity model\footnote{Entity models takes care the movements of a single node.} Gauss-Markov, which I from \cite{MobilityAdHocResearch} found to be a good algorithm for simulating a movement pattern\footnote{Implemented by the same team that build the BonnMotion movement tool} that seems rather random. For an example of the movment produced by the Gauss-Markov model please, see figure~\ref{fig:gauss-markov}\footnote{Full setup for the parameters that created it can be Gauss-Markov can be found in the \texttt{motion\_points} function in \texttt{src/Visualization/visualize\_graph.py}.}. 

Gauss-Markov works by for the $n$ step finding the next pair of coordiantes  ($x_n$, $y_n$) by finding a speed $s_n$ and direction $d_n$, based on the previous speed and direction $s_{n-1}$ and $d_{n-1}$, the previous coordinates $(x_{n-1}, y_{n-1})$, and a number of other constants. Due to space concerns I will not cite the equations here, but the interested reader can find them in section 2.5 in \cite{MobilityAdHocResearch}.

I would have liked to test the Disaster Area movement pattern found in \cite{disasterArea}, but I found it difficult to get the BonnMotion tool to generate the movement, as well as unsure on which definitions would create an interesting model for the different algorithms to be tested on\footnote{In order to create a disaster area movement simulation several different zones have to be defined -- e.g. a zone for the disaster, a zone for the triage, treatment and so on.}.

Other interesting Mobility Models for \acp{manet} that I did not get a chance to test includes the group mobility models such as Nomadic Community Mobility Model \cite{MobilityAdHocResearch} and the Pursue Mobility Model. Nomadic Community Mobility Model creates a reference point that moves, and the nodes make sure that they always in the area of it, while moving around using an entity mobility model. The Pursue Mobility Model models several nodes tracking a target, using a entity mobility model in order to ``randomize'' the nodes location.

The script I have used to create the needed ns-2 mobility data can be found in the \texttt{src/Motion/MotionCode} directory.

\subsubsection{Traffic}
\label{section:traffic_model}
I chose the TCP connection on the ground that I would like to stress test the different routing algorithms, and since TCP requires more back-and-forth than UDP (and is necessary for transfer of files, and not just steaming of video/audio or the state of a multi-player game).

Ns-2.33 comes with its own Traffic generation program cbrgen.tcl\footnote{Located in the \texttt{ns-2.33/indep-utils/cmu-scen-gen} directory in the ns-allinone-2.33 directory.}, that I have used to generate the traffic for these tests. The ns-2 documentation is very lacking in its description, but inspection of the code shows that it creates a connection uniformally between $0.0$ and $180.0$. I have however modified the script so that it can take any time. The modified version of the script can be found at \texttt{src/ns-2.33 changes/cbrgen.tcl}. The script I have used to create the needed ns-2 traffic files can be found in the \texttt{src/Motion/Traffic} folder.

In retrospect using a more advanced traffic generation utility would have been advisable, but there was little mention of it in the literature I found. I have however later found out that there exist a BonnTraffic\footnote{BonnTraffic can be found at \url{http://web.informatik.uni-bonn.de/IV/BoMoNet/} --- last accessed on the 21/8 2011.} tool, which I sadly could not get compiling in time. Another resource for traffic generation is found on the ns-2 wiki at \url{http://nsnam.isi.edu/nsnam/index.php/Contributed_Code#Topology_and_Traffic_Generation}\footnote{Last accessed on the 21/8 2011.}.

\subsubsection{Network simulators}
\label{section:network_simulators}

In order to perform a emperical evaluation, a network simulator is needed. I have chosen not to make my own, since it would take too much time, and would make it much harder to compare my results with other results. I have been able to identify four big network simulators: ns-2, ns-3, GloMoSim and the Opnet Modeler.

\begin{description}
\item[ns-2:] ns-2 is a discrete event network simulator. The latest version I have been able to find is version 2.34 from 2009. Ns-2 uses the programming language Otcl for most of its non-core engine files --- routing algorithms and the scenario files that configures how the test is to be performed. Otherwise everything is written in C++.
\item[ns-3:] ns-3 is the successor to ns-2 in an attempt to make a clean break from the architecture of ns-2 and its reliance on Otcl in favour for Python.
\item[GloMoSim:] GloMoSim is a parallel discrete-event simulator. GloMoSim is an old simulator, version 2.3, last updated December 2001\footnote{GloMoSim can be found at \url{http://pcl.cs.ucla.edu/projects/glomosim/academic/download.html} --- last accessed the 31/7 2011.}. GloMoSim uses the programming language \emph{Parsec}, a dialect of C, to write most of the non-core engine files -- such as routing algorithms. 
\item[Opnet Modeler:] Opnet is a commercial network simulator. Every layer of the network stack has to be modelled as a finite state machine \cite{MANcom}. 
\end{description}

From \cite{MANcom} it is clear that the network simulators gives widely different results. Since it is not clear which of the simulators which gives the result that is closest to reality, the parameters I will have to apply to find the best network simulator are: The license of the network simulator, when it was last updated, how configurable any given simulation is, whether or not it supports wireless routing, how much memory it consumes while running, which languages are required to implement a new protocol, how many programming languages are required to implement a new scenario, and whether new routing protocols can just be plugged in, or if the entire network simulator has to be recompiled. For configurability and memory requirements I will use the values A, B and C to grade them, where A is the best value and C is the worst.\\
\begin{scriptsize}
\begin{minipage}{15.0cm}
\begin{tabular}[4]{lllll}
                          & ns-2      & ns-3      & GloMoSim & Opnet Modeler \\
\hline
License:                  & GNU GPL   & GNU GPL   & Academic & People in academic world can apply\\        &           &           &          & for a monthly renewable license \\
Last updated              & 20/6 2011
\footnote{\scriptsize{See \url{http://sourceforge.net/projects/nsnam/} --- last accessed 31/7 2011.}} & May 2011
\footnote{\scriptsize{See \url{http://www.nsnam.org/ns-3-11/} --- last accessed 31/7 2011.}} & 19/12 2001
\footnote{\scriptsize{See \url{http://pcl.cs.ucla.edu/projects/glomosim/academic/download.html} --- last accessed 31/7 2011.}}  & \\
Configurability           & A         & A         & B         & Unknown, but I suspect A\\
Support for wireless      & Yes       & Yes       & Yes       & Yes \\
Memory requirements       & A         & B         & B         & Unknown\\
Req for routing protocol  & C++ and OTCL & C++    & Parsec    & None - uses built-in GUI\\
Req for new scenario      & OTCL      & Python    & Custom    & Unknown \\
Plug-in or Recompile      & Recompile & Recompile & Recompile & Plug-in, as it uses the built-in GUI
\end{tabular}
\end{minipage}
\end{scriptsize}

All of the simulators have simple node mobility models built-in. I have however chosen to use a third-party program, BonnMotion\footnote{BonnMotion has been produced as a collaboration between the University of Bonn and the Toilers at \url{toilers.mines.edu} -- last accessed the 16/5 2011.} on the grounds that it supports a far greater range of mobility models\footnote{Such as ``Disaster Area'', which I have not found built in implemented in any of the network simulators.}, and it enables me to store, and process, the movement traces.

Based on the above, I have chosen to use the ns-2 simulator, since it is far more customisable than both ns-3 and GloMoSim, is Open Source and referred to in much of the literature I have read \cite{directed, gpsr, energyConservation, two-tier} and is still being supported, unlike GloMoSim. However, while the 2.34 version is the latest, I have opted to use version 2.33 from 2008, since this is the only version I have been able to find that has a working copy of GPSR\footnote{Found at \url{http://www.cn.uni-duesseldorf.de/alumni/kiess/software/hls-ns2-patch} --- last accessed 31/7 2011.} with a location system. The location system in question is the \ac{hls} which borrows many concepts from the \ac{gls} \cite{hls}. Thanks to the working version of GPSR, I have been able to comparatively quickly implement both a Greedy routing algorithm and the GOAFR routing algorithm, which I will test amongst others. 

I have opted not to port the code to ns-2.34 on the grounds that I fear that I might introduce subtle bugs that might take a long time to find and correct. Since this is only one version behind the current stable release, I find this acceptable.

\subsubsection{Size of test}
In order to gain a better picture of the results I will perform each test 10 times, which in practice means that for each configuration of routing algorithm, simulation size, specific number of nodes there will be 10 tests. In total 1200 tests will be performed, each with their own unique movement and traffic trace\footnote{It is more important to test different combinations of movement and traffic than any run-time differences of the network simulator.}. While testing each configuartion of parameters only 10 times isn't that much, I have found that in practice the wall-clock time and hard-disk space requirements\footnote{A single test can take up to 440 MB}, this is a reasonable compromise.

\subsection{Limited-range spanner tests}
\label{section:test_desc_spanners}
In this section I will detail how I have chosen to set up the limited-range spanner tests for the three different graphs: \ac{ucg}, \acp{gabe} and \acp{rng}.

\subsubsection{Creation of the graphs}
Several graphs of different sixes must be created, as it otherwise is impossible to gain a proper perspective on how well the \ac{gabe} and the \ac{rng} stack up against the \ac{ucg}. In order to better describe the graphs, I introduce the notion of \emph{node-sets}, which is a finite set of nodes, where all nodes have a x- and y-coordinate and a transmit distance. Also of importence is the concept of \emph{node-category}, which is a set of node-sets, where all node-sets in any given node-category have the same amount of nodes in them. 

In order to do this test I first have to make several decisions:
\begin{itemize}
\item Decide upon whether we are going to uniformly distribute the nodes or if some sort of clustering should be done. 
\item The number of node-categories, balancing the need for data-points and their spread with the time they will take to create and analyse.
\item The density for the node-categories --- if indeed there should be a fixed density for them, or if the density should be random.
\item The number of graphs and the number of tests for each graph in order to get enough test data to properly gauge the underlying data.
\end{itemize}

I have not encountered any specific point clustering algorithm in the literature besides the various mobility models (see \cite{disasterArea, MobilityAdHocResearch}) which has lead me to place the nodes uniformly in the plane. This was done out of interest of how the \ac{ucg}, the \ac{gabe} and the \ac{rng} would behave in generel, and not just in \acp{manet}. In retrospect using node-sets constructed using different  mobility models have given a better picture, and would most likely have cut down on the number of node-sets and graphs that would have had to be generated --- See section~\ref{section:future_work} for my recommendation for future comparisons between the graphs.   
Since I have chosen to uniformly distribute the nodes I will need a large number of tests in order for the underlying properties of the three types of graphs to emerge. Therefore I will therefore have to limit the number of node-categories so that I will be able to process all the tests. I will therefore chose to limit myself to 9 node-categories with the following number of nodes: 50, 100, 250, 500, 1000, 2500, 5000, 7500 and 10000 nodes.

This is a good compromise between having a reasonable number of data-points and big enough spread, while still being able to compute them in a reasonable amount of time. The nodes are going to be placed uniformly in the plane, with the lower bound of their coordinates being 0, and the upper bound depending on the number of points in the graph. No two points will have same the coordinates. 

Since we are dealing with limited-range spanners (See section~\ref{section:spanners}), the value of the node density and the radio-range of the nodes becomes important\footnote{If I had made the choice to use the mobility models or a cluster node algorithm, these factors would still have an impact, but properly less.}. Different densities and radio-ranges would heavily influence the outcome of the results, as a high node density/radio-range would favour the \ac{ucg} over the two other graph types, while a low density/radio-range would create heavily disconnected graphs in all three cases. Since having many disconnected graphs would make testing difficult, I have tried to find a combination that would give me a low number of connected components pr. node-set, while still keeping the density reasonably low. From several experiments this let me to a density of 0.01 nodes pr. square unit and a radio range of 20 units, and make the space the nodes are placed over a function of the number of nodes. Specifiably, if we let $x$ be the length of the sides in a square and $num$ be the number of nodes, then we have that the density is
\eqn{
\frac{num}{x^2} = 0.01 \Leftrightarrow x = \sqrt{100 * num}  
}
meaning that in practise the graphs are squares in the plane with the following dimensions, which will also be the upper limits on their coordinates:\\
{\tiny
\begin{tabular}{lrrrrrrrrr}
Number of nodes: & 50 & 100 & 250 & 500 & 1000 & 2500 & 5000 & 7500 & 10000 \\
Width/height:    & 71 & 100 & 158 & 223 &  316 &  500 &  707 &  866 & 1000 
\end{tabular} 
}

\subsubsection{Tests to be performed}
\label{section:spanner_tests_performed}
In order to check the limited-range spanner for the three graphs, I will randomly pick two different nodes, a \emph{node-pair}, that are connected in the \ac{ucg} and record the shortest number of hops (the \emph{Unit distance}), as well as the shortest distance based on the weight between the two nodes (the \emph{Euclidean distance}). Both of these values are worth finding, as the number of hops is directly applicable to \acp{manet}, while the Euclidean distance is interesting from a more theoretical perspective, and will therefore also be recorded.

I will do this for all three types of graphs, and record whether or not a path exist between the node-pair in the \ac{gabe} or the \ac{rng}. If it is indeed the case that there are cases where the two alternative planar graphs removes paths, that would create an element of risk by employing them, instead of the \ac{ucg}. On these grounds I will record and display the number of missing paths the analysis is able to find.

In both cases I will use Dijkstra's algorithm to find the smallest Euclidean/Unit distance. The implementation I uses is a modified version originally made by David Eppstein\footnote{dijkstra.py was found at \url{http://code.activestate.com/recipes/119466-dijkstras-algorithm-for-shortest-paths/} --- last accessed 17/8 2011. Modification was to allow to check for Unit distances without having to modify the graph weights -- see function ``UnitDijkstra''. Support library prirodict.py (also by David Eppstein, and not modified) was found at \url{http://code.activestate.com/recipes/117228/} --- last accessed 17/8 2011. The files can be found in \texttt{src/graph\_support}.}.

\subsubsection{Analysis}
For each node-pair tested for the graphs the total distance traversed between all node-pairs, the average distance, the average maximum and minimum distance for all the tests\footnote{The average value is needed since there may be outliers that easily could skew the results.} and lastly the standard deviation\footnote{For the same reasons as I included the average minimum and maximum.}. Likewise I will find the total Unit distance, as well as the average, average maximum and average minimum Unit distance and standard deviation. Since one of the advantages of the \ac{gabe} and the \ac{rng} is that they have fewer neighbours, the average, max- and minimum number of neighbours and standard deviation will be recorded. The average number of connected components (across all the 500 \acp{ucg}) will also be recorded and represented, since this the number is a good indicator of how connected the graphs are, and therefore the quality of the results.

\subsubsection{Size of the test}
Since the node-sets that we are going to generate the graphs from are going to distributed uniformly, there is a need to generate many node-sets, so that the underlying patterns can emerage. Therefore I have decided to generate 500 different uniformly distributed node-sets. For each node-set I will then generate one \ac{ucg}, one \ac{gabe}, and one \ac{rng}, meaning that for each node node-category I  generate 500 graphs for the \ac{ucg}, \ac{gabe} and \ac{rng} (for a total of 1500 graphs for each. Each node-set is then used to generate the three types of graph. For each node-set I then generate 100 node-pairs. This gives a reasonable amount of data to analyse, while still being computable for the larger number of nodes.
