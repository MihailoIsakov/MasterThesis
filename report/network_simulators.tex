\section{Network simulators}

In order to test any algorithm I might implement, I will need a network simulator, to check how good it is compared to already existing algorithms. I have been able to identify 3 big network simulators: NS-2, GloMoSim and the Opnet Modeler.

\begin{description}
\item[NS-2:] NS-2 is a discrete event network simulator. The last version I have been able to find is from 2009. NS-2 uses the programming language Otcl for most of its non-core engine files -- such as routing algorithms and the scenario files that configures how the test is to be performed.
\item[NS-3:]
\item[GloMoSim:] GloMoSim is a parallel discrete-event simulator. GloMoSim is an old simulator. I have not been able   . GloMoSim uses the programming language \emph{Parsec}, a dialect of C, to write most of the non-core engine files -- such as routing algorithms. 
\item[Opnet Modeler:] Opnet is a commercial network simulator. Every layer of the network stack has to be modelled as a finite state machine \cite{MANcom}.
\end{description}

From \cite{MANcom} it is clear that the Network simulators gives widely different results. Since it is not clear which of the simulators which gives the result that is closest to reality, the parameters I will have to apply to find the best network simulator are: How many other articles use the network simulator\footnote{Since this means I would been able to compare it to more articles}, how flexible is it, how difficult is it to configure scenarios and how difficult is it to write new routing algorithms.

\begin{tabular}[3]{l|c|c}
                          & NS-2      & GloMoSim \\
\hline
License:                  & GNU GPL   & Academic \\
Cost                      & Free      & Free for educational use \\
Last updated              & 2011-05-1\footnote{See \url{http://sourceforge.net/projects/nsnam/}} & 2001-19-12\footnote{See \url{http://pcl.cs.ucla.edu/projects/glomosim/academic/download.html}}\\
Configurability           & Extensive & Sligtly less extensive \\
Support for wireless      & Yes       & Yes \\
Memory requirements       & Very high & High \\
Req for routing protocol  & C++ and OTCL & Parsec \\
Req for new scenario      & OTCL      & \\
\# of different citations & & \\
Plug-in or Recompile      & Recompile & Recompile \\

\end{tabular}\\

Most of the articles I have read that have named their network simulator uses NS-2 [Find articles and cite them].

Both NS-2 and GloMoSim are very configurable, but NS-2 is more flexible than GloMoSim and has both more built-in features.

From what I have been able to derive from the manuals, and examples from both NS-2 and GloMoSim, GloMoSim is less complicated than NS-2.

Comparing NS-2 and GloMoSim it is clear that NS-2 is both the more complicated program to set up and program for. It is 

Both NS-2 and GloMoSim are free for education, while I will have to pay for Opnet.

Both NS-2 and GloMoSim has simple mobility models for the nodes built in. However, I have chosen a separate program, BonnMotion, which has been produced as a collaboration between the University of Bonn and the Toilers at \url{toilers.mines.edu} \todo{find a date where you last accessed it and write it -- perhaps this should be moved}. The program has multiple mobility methods built in -- such as ``Random Waypoint'', ``Manhatten Grid'' and 2 different
