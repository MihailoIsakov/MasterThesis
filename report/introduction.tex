\section{Introduction}
\label{section:introduction} 

\subsection{Motivation}

In the last few years we have seen a clear increase in the sale of mobile phones with an advanced OS, Internet access, GPS and large amounts of computing power. These so--called \emph{Smart Phones} are becoming increasingly common, replacing ``dumb'' mobile phone in the developed world. Since third-party developers are encouraged to write custom applications to these small mobile computers, a small cottage industry has sprung up around them.
 
Many of these Smart Phones has a built-in GPS or WiFi triangulation tools, which can be accessed by the phones applications, making it possible to create services that uses the users location data --- such as directional services (for instance Google Maps) or social networking services (Foursquare, Facebook, Google+ and Twitter).

Given that these small portable computers with broadcast and receiving capabilities are now a reality, it would be interesting to see if these devices could form a local network, wherein messages, without the need for any kind of centralised infrastructure, could be routed. Since we no longer have the luxury of a centralised infrastructure or service, the network would have to be created and maintained on the fly by the Smart Phones. Having mobile devices, from now on referred to as \emph{nodes}, create such a network is the study of the field called \ac{manet}. which itself is a off-shot of the field of \ac{anet}, which only deals with stationary devices. Practical applications of these are sometimes called \emph{Mesh networks} because the coverage of the network is created from a patchwork of devices -- a mesh, each listing and transmitting data, without the need for dedicated network devices.  

In \acp{anet} research energy conservation is usually a high priority. Since \acp{anet} have mostly been deployed in inhospitable and inaccessible regions, which often makes it infeasible to recharge or replace the node's power source, this makes sense. However, recharging portable computers in the developed world is a trivial task, and power conservation will therefore not have a high priority in this thesis. 

A practical use of \acp{manet} could be in humanitarian operations, such as after a earthquake, flood, hurricane etc. In such situations the infrastructure tends to be heavily damaged, and there will be a great need for coordination to maximise relief efforts. Imagine for instance that there has been an earthquake that has destroyed the wired infrastructure in a town. It is clear, that in order to the most good, the relief workers will need the ability to coordinate their efforts, otherwise some of the work is going to be duplicated, and things that needed immediate attention might be overlook for other, less urgent, problems. Since the local wired infrastructure has been destroyed, the relief workers cannot connect to the local Internet or telephone network. The relief workers may have brought in a \ac{wbs} (with some kind of long-lasting power source), which gives them access to the Internet. However, such a station is likely to be very expensive and bulky, and it is therefore infeasible to scatter these around the entire town. While it is almost a given that devices in its vicinity can connect to the Internet through WiFi, and the \ac{wbs} can improve the distance by boosting the signal, there is a upper limit to its range, before the noise-to-signal ratio becomes too great. See figure~\ref{fig:wbs1} for an illustration of this. An investigation into the movement patterns in such a situation is done in \cite{disasterArea}, which I will use in this report.

\tikfig{wbs-illu}{In figure~\ref{fig:wbs1} we see a network with one \ac{wbs} and a number of nodes and their transmission radius. All nodes, besides the \ac{wbs} (which is the only node that is connected to the Internet), are functionally alike, but have different colours and shapes depending on their level of connection to the \ac{wbs}. The grey nodes have full connection with the \ac{wbs}, while the striped nodes can receive information from the \ac{wbs}, but not transmit information to it. The black nodes can neither transmit nor receive any data to or from the \ac{wbs}. In figure~\ref{fig:wbs2} forwarding stations have been added, allowing all nodes in the network to be connected. In figure~\ref{fig:wbs3} we have made an (Mobile) Ad-hoc network, connecting all nodes but the bottommost diamond.}

An alternative could be to manually create a wired network, but this can be very time consuming, and prone to disruption as people or vehicles may destroy the connection. A third alternative may be to use \acp{wfs}, which must be placed such that they can both send and receive information from the \ac{wbs} and in this way  make it possible to extend the area that can receive wireless Internet --- see  figure~\ref{fig:wbs2} for an illustration of this. However, these \acp{wfs} are still an extra burden for the relief workers, and it is unlikely that they will be able to cover every knock and cranny. One way to reduce the need of the \acp{wfs} is to make each node in the network propagate the wireless Internet signal. For the system to be of any use, the topology would have to be created on the fly. See figure~\ref{fig:wbs3} for an illustration of a \ac{manet} in the case of the humanitarian example.

Alternative applications include:
\begin{itemize}
\item Taking pressure off the cellular network during disasters, when everybody is trying to reach their loved ones.
\item Help third world nations bring wireless Internet (which from an infrastructure point of view is cheaper than cables) to their citizens by reducing the number of WiFi towers needed, by letting individual devices propagate the information. More controversial it could be used to create a decentralised network independent of the telecommunications companies, that would allow them to communicate even if the authorities shut down the Internet. 
\item Allow autonomous robot swarms to communicate and share information without the need for a centralised service. In the future self driving cars could automatically detect traffic congestion or alert the cars around it that it has malfunctioned. 
\end{itemize} 

There are numerous aspects to create a proper \ac{manet} or \ac{anet}, but two of the major topics of research are how to create the network topology, and how to route the messages, so that the network maximises its lifetime (since nodes will fail when their power supply is exhausted). Unsurprisingly designing a proper \ac{manet} is more difficult than a \ac{anet}, since nodes can join and leave a \ac{manet} with an ease that is not possible in an \ac{anet}. 

\subsection{Expectations to the reader}
I expect the reader to have a university level understanding of Computer Networking\footnote{For instance having read \cite{ComNet} or similar.}, especially when it comes to understanding the building blocks of the network stack, as well as the most common problems with wireless networks. I also expect the reader to have an university level understanding of algorithms and graph theory.

\subsection{Scope and limitations}
While important in a real world application, I will in this thesis not concern myself overly much with security, neither in the form of the authentication, anonymity of the sender or ensuring that the individual nodes are not misbehaving, or behaving maliciously. For a treatment of security, see \cite{trustedRouting}. Furthermore, I will not try to solve the problem of sending a message from a source to a sink if they cannot be connected by a path.I will furthermore assume that all nodes have the same broadcast radius unless explicitly specified.

\subsection{Anatomy of a \ac{manet}}
A real-life implementation of a \ac{manet} will need the following elements:
\begin{description}
\item[Routing protocol:] The routing from of a message from a source to the sink.
\item[Routing graph:] The logical graph that the messages are routed through.
\item[Connectivity graph:] The underlying graph that the routing graph is a subset of.
\item[Transport protocol:] The protocol that sends messages between nodes. In practise this means TCP, UDP or a custom protocol.
\end{description}

The above points are, however, only the broad outline of the major features that have to be implemented for a \ac{manet} to work. The following list of problems will also have to be taken care of in order to have a fully functioning \ac{manet}:

\begin{description}
\item[Entering and Merging networks:] When a node enters a new network, it needs to inform the network of its presence. Furthermore, when the node is already part of a secondary network (none of which are in the first network), then the networks need to be merged together.  
\item[Coordinate scheme:] Most routing schemes uses coordinates in order to find an efficient path, for any given metric, to a node. Therefore a fully working solution must decide on how the the nodes will represent their location. Two options include giving each node their own real world coordinates (though this requires extra power dedicated to the needed GPS or WiFi), or have their location be relative to the other nodes (with the distance being calculated through signal strength and triangulation \cite{geoNoInfo}.
\item[Node location broadcast:] No matter what kind of location scheme the nodes are given, they will need to have a way to keep their location up to date. The two main choices are here whether the nodes will periodically send out location messages, or if they will only have an initial deployment, and after that piggyback this information with their normal traffic.
\item[Node location mechanism:] There needs to be a mechanism so that a node can retrieve the coordinates of the node it wants to come into contact with. One solution to this would be that each node contained a table with all the nodes location, (as is the case for the routing algorithm DSDV\footnote{See section~\ref{section:cat_routing} on p.~\pageref{section:cat_routing}.} \cite{DSDV}), or 
\item[Hierarchy:] Whether or not the nodes are going to route through levels of so-called \emph{cluster nodes} that will service numerous nodes.
\end{description} 

In this thesis I have decided to focus on the routing protocol and the routing graph, but will touch briefly on the other problems as they become relevant to these issues..

\hide{
\subsection{Background Material Used}
For this project I have read the following articles for background information: ``A Survey on Wireless Mesh Networks'' \cite{martinWirelessSurvey}, ``Mesh networks: commodity multihop ad hoc networks''\cite{martinMeshnetwork} \todo{continue to include the articles you have read and are useful, but that have not been included directly}
}

\subsection{Source code and test data}
\label{section:source_code}
The source code I have written myself (or modified, and not covered by another license), report and some of the test data can be found in my GitHub repository at \url{https://github.com/kjmikkel/MasterThesis} and is released under the GPL v2.

The python scrips I used to generate and tests the graphs can be found at \texttt{src/spanner/spanner\_check.py} along with the data (see the README for explanation) and \texttt{src/spanner/graph\_support} All requirements in the code is mentioned in the README file.\\

Due to the sheer size of the test data (well in excess of 10 GB), I am unable to upload all of it to my repository, and I have therefore only uploaded the node-sets, since all the graphs will be able to be created from them (though sadly not the node-pairs, as they too take up too much space).

\subsection{Overview}
In this section I give a quick overview of what the different sections in report will cover.
\begin{description}
\item[Section \ref{section:introduction}] The introduction to the report, which will introduce the problem of Mobile Ad-Hoc Networking.
\item[Section \ref{section:fundamental_graph_routing}] Description of the fundamental concepts of graph routing as well as an introduction of the definitions and concepts that I will use throughout the report.
\item[Section \ref{section:routing_graph}] Introduction of the more advanced graph concepts as well as the graph types I will use in the network simulation.
\item[Section \ref{section:routing_algorithms}] Description of some of the many types of routing algorithms that exist, as well as the many problems a \ac{manet} poses.
\item[Section \ref{section:test_description}] Discussion of which graph algorithms are to be used and which combinations of parameters are to be tested, as well as the choice of network simulator.
\item[Section \ref{section:test_results}] Presentation of the test results
\item[Section \ref{section:discussion}] Discussion of the test results
\item[Section \ref{section:conclusion}] Conclusion and wrap-up.
\end{description}
