\documentclass[A4, 12pt, english]{article}
\usepackage{mikkel}
\usepackage[ruled, vlined]{algorithm2e}
\input comment.sty
\excludecomment{planar-footnote}

\title{Energy Affluent Mobile Ad-Hoc Networks}
\author{Mikkel Kj\ae r Jensen}
\date{\today}
\begin{document}
\pagestyle{fancy}

% My acronyms
\acrodef{ch}[CH]{Cluster Head}
\acrodef{ucg}[UCG]{Unit Complete Graph}
\acrodef{gabe}[GG]{Gabriel Graph}
\acrodef{udg}[UDG]{Unit Disk Graph}
\acrodef{manet}[MANET]{Mobile Ad-Hoc Network}
\acrodef{anet}[ANET]{Ad-Hoc Network}
\acrodef{rng}[RNG]{Relative Neighbourhood Graph}
\acrodef{rdg}[RDG]{Restricted Delaunay Graph}
\acrodef{cldp}[CLDP]{Cross-Link Detection Protocol}
\acrodef{wbs}[WBS]{Wireless Base-Station}
\acrodef{wfs}[WFS]{Wireless Forwarding-Station }
\acrodef{uga}[UGA]{Unit Graph Assumption}
\acrodef{hls}[HLS]{Homezone Location Service}
\acrodef{gls}[GLS]{Grid Location Service}

\maketitle

\clearpage

\abstract{
A \ac{manet} is a network of moving nodes that without human intervention can create its own network topology and ensure successful routing of messages between nodes. 

In this thesis I will give an overview of the elements required to create a \ac{manet} and give a short survey of several already existing solutions that can be used  

Furthermore, I will give a performance evaluation of my own implementation of the GOAFR algorithm \cite{gopher}, with I will empirical compare to the GPSR \cite{gpsr}, Greedy \cite{gopher, beyondUnit} and DSDV \cite{DSDV} routing algorithm using the ns-2 network simulator.

I also perform an empirical comparison of the number of neighbours and both the Euclidean and Unit distance between two randomly chosen nodes in the \emph{\ac{ucg}} (graphs where two nodes are connected or not based on their distance, regardless of whether they make the graph non-planar or not), the \emph{\ac{gabe}}, and the \emph{\ac{rng}}. This comparison shows that the \ac{gabe} and \ac{rng}, for uniform placed nodes, has a behaviour similar to that of a spanner, and that they on average will have 3.6 and 2.5 neighbours.
}

\clearpage
\section*{Acknowledgements}
I would like to thank my adviser Martin Zachariasen for his guidance, advice and encouragement in writing this thesis.

I would like to thank the Toilers group for making the BonnMotion Mobility trace generator available to the public.
\clearpage

\tableofcontents

\cinclude{introduction} 
\cinclude{fundamentals}
\cinclude{graph_overview}
\cinclude{routing}
\cinclude{test_description} 
\cinclude{test_results} 
\cinclude{discussion} 
\cinclude{conclusion}

\appendix
\cinclude{appendix}

\bibliographystyle{plain}
\bibliography{references}{}

\end{document}
