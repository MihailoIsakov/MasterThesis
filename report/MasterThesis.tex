\documentclass[A4, 12pt, english, draft]{article}
\usepackage{mikkel}
\usepackage[ruled, vlined]{algorithm2e}
\input comment.sty
\excludecomment{planar-footnote}

\title{Energy Affluent Mobile Ad-Hoc Networks}
\author{Mikkel Kj\ae r Jensen}
\date{\today}
\begin{document}
\pagestyle{fancy}

% My acronyms
\acrodef{ch}[CH]{Cluster Head}
\acrodef{gabe}[GG]{Gabriel Graph}
\acrodef{udg}[UDG]{Unit Disk Graph}
\acrodef{manet}[MANET]{Mobile Ad-Hoc Network}
\acrodef{anet}[ANET]{Ad-Hoc Network}
\acrodef{rng}[RNG]{Relative Neighbourhood Graph}
\acrodef{rdg}[RDG]{Restricted Delaunay Graph}
\acrodef{cldp}[CLDP]{Cross-Link Detection Potocol}
\acrodef{wbs}[WBS]{Wireless Base-Station}
\acrodef{wfs}[WFS]{Wireless Forwarding-Station }
\acrodef{uga}[UGA]{Unit Graph Assumption}
\acrodef{hls}[HLS]{Homezone Location Service}
\acrodef{gls}[GLS]{Grid Location Service}

\maketitle

\abstract{
A \ac{manet} is a network of moving nodes that without human intervention can crate its own topology and ensure successful routing of messages between nodes. 

In this thesis I will give an overview of the elements required to create a \ac{manet} and several already existing solutions that fulfils these requirements. 

Furthermore I will give a performance evaluation of my own implementation of the GOAFR algorithm\cite{gopher} compared to several other algorithms. I will also perform a comparison between the average length of two nodes in randomly generated non-planar graphs, compared to the \ac{gabe}\   and the \ac{rng} \cite{RNG}, compared to the average number of neighbours and edges in the graph.
\todo{Improve abstract}
}
\listoffixmes
\tableofcontents



%\cinclude{introduction} 
%\cinclude{fundamentals}
%\cinclude{graph_overview}
%\cinclude{routing}
%\cinclude{network_simulators} 
%\cinclude{test_description} 
\cinclude{test_results} 
%\cinclude{discussion} 
%\cinclude{conclusion}

\appendix
\cinclude{appendix}

\bibliographystyle{plain}
\bibliography{references}{}

\end{document}
